
\begin{abstract}

Este trabajo consiste en la definición y estudio de
tres lenguajes de programación. Los dos primeros serán 
lenguajes funcionales, uno con un sistema de tipos simple 
y otro con un sistema de tipos que soporta subtipado. El
tercero es un lenguaje funcional con aspectos imperativos,
perteneciente a la clase de lenguajes Algol-Like. 

Para la definición semántica se utiliza teoría de categorías,
en particular en la definición de los modelos semánticos. En
particular, siguiendo propuestas de Reynolds y Oles, utilizamos 
categorías funtoriales para el lenguaje Algol-like.

Además se presentan las pruebas de ciertas propiedades 
deseables de las modelos semánticos dados: para el primer
lenguaje nos enforcamos en la continuidad de las ecuaciones  
semánticas y en la corrección de la reducción;en el segundo
lenguaje, desarrollamos la prueba de coherencia para diferentes
derivaciones del mismo juicio; y para el tercero, probamos la
naturalidad de las ecuaciones semánticas.

El trabajo teórico estuvo acompañado de la implementación
de evaluadores en Idris, un lenguaje con tipos dependientes.
Este desarrollo nos permitió descubrir algunas dificultades
en ciertas ecuaciones semánticas.

\end{abstract}

\null
\vfill

\textbf{Clasificación:}\\

F.3.2 - Semantics of Programming Languages - Denotational semantics.\\
\indent
F.4.1 - Mathematical Logic - Lambda calculus and related systems.\\

\textbf{Palabras clave:}\\

Semántica denotacional, Categorías, Categoría funtorial, Stack discipline.\\
