\chapter{Tipos}
\label{chap:types}

En la introducci\'on ya comentamos algo sobre tipos, en particular la 
motivaci\'on de contar con ellos. En este cap\'itulo vamos a estudiar las 
nociones importantes sobre tipado, a su vez, vamos a tener un primer contacto
con la idea de sem\'antica denotacional. \

Es importante comentar que los tipos que estudiemos en este cap\'itulo van
a formar parte de los tipos concretos que usemos a lo largo de los dem\'as
cap\'itulos.


\section{Tipos simples}

\subsection{Sintaxis}
Comencemos introduciendo tipos b\'asicos que representen los tipos de datos,
una vez con estos definidos, podemos definir los tipos b\'asicos para las
frases de nuestro lenguaje y agregar operadores para en base a estos
construir nuevos.\

Una pregunta a esta altura puede ser, ?`Por que tener dos clases de tipos 
distinta?; de momento puede no tener sentido esto, pero simplemente
comentamos que por el m\'etodo de evaluaci\'on de nuestro lenguaje
vamos a querer esta distinci\'on.\

Definamos, ahora si, la sintaxis de los tipos de dato b\'asicos, 
que vamos a llamar $data$ $types$ y la de los tipos de las frases que
llamaremos $phrase$ $types$ 

\begin{center} $\langle data \ type \rangle ::=$ ${\ int}$ $|$ ${\ real}$ $|$ ${\ bool}$ \end{center}

\begin{center} 
$\langle phrase \ type \rangle ::=$ $\delta {\ exp}$ $|$ $\delta {\ acc}$ $|$ $\delta {\ var}$ $|$ ${\ comm}$ \\

\ \ \ \ \ \ \ \ \ \ \ \ \ \ \ \ \ \ 
\ \ \ \ \ \ \ \ \ \ \ \ \ \ \ \ \ \ 
\ \ \ \ 
$|$ $\langle phrase \ type \rangle$ $\rightarrow$ $\langle phrase \ type \rangle$
\end{center}

donde $\delta$ $\in$ $\langle data \ type \rangle$.

De momento introducimos solamente el operador $\rightarrow$, adelante en este 
cap\'itulo vamos a agregar otros mas e iremos viendo como se completa nuestro
sistema de tipos.\

\subsection{Inferencia}
Para definir la inferencia de tipos primero vamos a necesitar introducir las 
nociones de contexto y juicio de tipado. Pero antes, veamos a que nos referimos
con inferencia de tipos, la idea es que dado que ciertas expresiones tiene 
su correspondiente tipo poder inferir el tipo de expresiones mas complejas.
Por ejemplo, nuestro futuro lenguaje va a contener el operador "+" cuyo significado
es el usual, un posible tipo podr\'ia ser ${\ intexp} \rightarrow {\ intexp} \rightarrow {\ intexp}$
por lo tanto si tenemos la expresi\'on $v + w$, donde $v,w$ son variables en
nuestro lenguaje, vamos a querer poder inferir que los tipos de $v$ y $w$ son
${\ intexp}$.\

\

\

\noindent Definamos contexto y juicio de tipado, un contexto sera una lista de 
patterns y $phrase types$

\begin{center} 

$\langle context \rangle ::=$ 
		$\langle context \rangle , \langle patterns \rangle:\langle phrase \ types \rangle$

\end{center}
(no me acuerdo bien que eran los patterns, completar ac\'a)

ahora entonces un juicio de tipado sera, una relaci\'on entre un contexto $\pi$ y un expresi\'on $e$
con su determinado tipo $\theta$, lo denotamos como

\begin{center} $\pi \vdash e : \theta$ \end{center}

y decimos que $e$ tiene tipo $\theta$ mediante $\pi$.\

Ahora estamos en posici\'on de definir la inferencia de tipos mediante reglas,

\

(ac\'a van las reglas de inferencia de tipos escritas en las slides, creo que por primera vez
hace falta conocer la sintaxis del lenguaje.)

\subsection{Sem\'antica denotacional}

(La sem\'antica definida en las slides pero sin considerar stack discipline.)


\section{Sub-tipos}

\section{Intersecci\'on}
