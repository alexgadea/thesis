\chapter{C\'alculo lambda tipado con subtipos}

En el cap\'itulo anterior definimos la sint\'axis y sem\'antica del lenguaje que llamamos
$\lambdaarrow$, para esto en el transcurso introducimos distintos conceptos sobre el tipado.
Ahora vamos estudiar a $\lambdaleq$, el cual tiene como fin introducir conceptos sobre
el subtipado, por esta raz\'on mantenemos el lenguaje exactamente igual al lenguaje 
del capitulo anterior. La principal diferencia entonces estar\'a en la definici\'on
de la categor\'ia de tipos y en que vamos a agregar una nueva forma de juicio de tipado
as\'i como nuevas reglas.\\

Entre los tipos del lenguaje $\lambdaarrow$, y por lo tanto los de $\lambdaleq$, tenemos
a $\intt$ y $\realt$ los cuales representan los conjuntos de enteros y reales matem\'aticos.
Algo interesante a pensar es que los enteros forman parte de los reales; es decir,
$\Z \subseteq \R$, luego surge la pregunta: ¿Existir\'a una forma de expresar esta relaci\'on 
como una relaci\'on entre los $\lrangles{Type}$?. La respuesta es s\'i y es el subtipado.

\section{Sintaxis para el subtipado}

Comencemos introduciendo la nueva forma de juicio de tipado, esta axiomatiza la relaci\'on
de que $\theta$ es subtipo de $\theta'$ cuando $\theta \leq \theta'$.\\

Primero veamos reglas de inferencia generales a cualquier sistema de tipos, empecemos discutiendo
una idea intuitiva de las reglas que ser\'ian deseables. Supongamos tenemos una frase $e$ con tipo
$\theta$ y adem\'as tenemos que $\theta' \leq \theta$ entonces podemos probar $e$ tiene tipo
$\theta'$.\\

\noindent
$\texttt{Ty Rule:}$ Subsumption.

\begin{center}
\AxiomC{$\pi \vdash e : \theta$}
\AxiomC{$\theta \leq \theta'$}
\BinaryInfC{$\pi \vdash e : \theta'$}
\DisplayProof
\end{center}

Supongamos tenemos que la expresi\'on $e$ tiene tipo $\intt$ 
y adem\'as que $\intt$ es subtipo de $\realt$, luego quisi\'eramos 
poder decir que $e$ tiene tipo $\realt$, adem\'as si suponemos un tipo $\natt$ que es
subtipo de $\intt$, entonces deber\'iamos poder decir que $\natt$ es subtipo de $\realt$,
es decir, tener transitividad entre los tipos, cualquier tipo es
subtipo de \'el mismo, es decir, los tipos son reflexivos. Con esto vemos que la relaci\'on
de subtipado es un preorden. \\

\noindent
$\texttt{Ty Rule:}$ Trans.

\begin{center}
\AxiomC{$\theta \leq \theta'$}
\AxiomC{$\theta' \leq \theta''$}
\BinaryInfC{$\theta \leq \theta''$}
\DisplayProof
\end{center}

\noindent
$\texttt{Ty Rule:}$ Reflex.

\begin{center}
\AxiomC{}
\UnaryInfC{$\theta \leq \theta$}
\DisplayProof
\end{center}

Para finalizar, supongamos tenemos $\theta_0 \leq \theta_0'$ y $\theta_1 \leq \theta_1'$ y adem\'as
que tenemos una expresi\'on $e$ que tiene tipo $\theta_0' \rightarrow \theta_1$. Luego
$e$ puede aplicarse a elementos de tipo $\theta_0$ y el resultado de tal aplicaci\'on
puede ser un elemento de tipo $\theta_1'$.\\

\noindent
$\texttt{Ty Rule:}$ Func.

\begin{center}
\AxiomC{$\theta_0 \leq \theta_0'$}
\AxiomC{$\theta_1 \leq \theta_1'$}
\BinaryInfC{$\theta_0' \rightarrow \theta_1 \leq \theta_0 \rightarrow \theta_1'$}
\DisplayProof
\end{center}

Estas reglas que mencionamos tiene la particularidad de ser generales para
cualquier sistema de tipado, definamos ahora mas reglas en relaci\'on a nuestros
tipos y lenguaje concreto, esto ser\'a definir la relaci\'on entre enteros y reales
y vamos a agregar una mas, tal vez no sea lo m\'as recomendado, por cuestiones sem\'anticas,
en cuanto al diseño del lenguaje pero es practico considerarla, para tener casos simples
de transitividad, que es la relaci\'on entre booleanos y enteros.\\

\noindent
$\texttt{Ty Rule:}$ boolToint.

\begin{center}
\AxiomC{}
\UnaryInfC{$\boolt \leq \intt$}
\DisplayProof
\end{center}

\noindent
$\texttt{Ty Rule:}$ intToreal.

\begin{center}
\AxiomC{}
\UnaryInfC{$\intt \leq \realt$}
\DisplayProof
\end{center}

\section{Sem\'antica para $\lambdaleq$}

Ahora que tenemos definido la nueva forma de juicio de tipado para la relaci\'on entre los 
tipos, vamos a actualizar nuestra categor\'ia de tipos, esta dejar\'a de ser
una categor\'ia discreta y la clave est\'a en que ahora nuestra relaci\'on entre
los tipos determinar\'a las flechas. Antes vimos que la relaci\'on que surg\'ia entre
los tipos daba lugar a un preorden entre estos, luego nuestra categor\'ia de tipos
ser\'a este preorden visto como categor\'ia.

\begin{definition}\label{lambdal:typescategory}
La categor\'ia de tipos, que nombraremos $\Theta$, se define como sigue\\

$\Theta_0$ $=$ $\{\theta \ | \ \theta \ \in \lrangles{Type} \}$\\
\indent
$\Theta_1(\theta,\theta')$ $=$ $\{\theta \rTo^{*} \theta' \ | \ \theta \leq \theta'\}$

\end{definition}

El preorden entre los tipos nos implica que dados
dos tipos $\theta$ y $\theta'$, $| \ \Theta_1(\theta,\theta') \ | \leq 1$.

Adem\'as esta nueva definici\'on nos impacta directamente en el
funtor sem\'antico $\semBrcks{ \ } : \Theta \rightarrow \CD$, ahora tenemos que 
definir como act\'ua el funtor con respecto a las flechas.

\begin{definition}\label{lambdal:typesemfunctor}22
Sea $\semBrcks{ \ } : \Theta \rightarrow \CD$ un funtor, tal que\\

$\semBrcks{\delta}_0$ $=$ $(S_\delta)_\bot$\\
\indent
$\semBrcks{\theta \rightarrow \theta'}_0$ $=$ $\semBrcks{\theta'}_0^{\semBrcks{\theta}_0}$\\

\[
\semBrcks{\boolt \leq \intt}_1 \ x =
\begin{cases}
0  & \text{si } \text{x} \\
1  & \text{si } \neg \text{x}
\end{cases}
\]
\indent
$\semBrcks{\intt \leq \realt}_1$ $=$ $\J$ \ \ \ donde $\J$ la inyecci\'on de enteros en reales.\\
\indent
$\semBrcks{\theta \leq \theta}_1$ $=$ $1_{\semBrcks{\theta}}$\\
\indent
$\semBrcks{\theta \leq \theta''}_1$ $=$ $\semBrcks{\theta' \leq \theta''}_1 \circ \semBrcks{\theta \leq \theta'}_1$\\
\indent
$\semBrcks{(\theta_0 \rightarrow \theta'_0) \leq (\theta_1 \rightarrow \theta'_1)}_1$ 
				$=$ 
				$\lambda f \in \semBrcks{\theta_0 \rightarrow \theta'_0}_0$ .
				$\semBrcks{\theta'_0 \leq \theta'_1}_1 \circ f \circ \semBrcks{\theta_1 \leq \theta_0}_1$\\

\end{definition}

Antes de seguir con la nueva definici\'on de la categor\'ia de contextos analicemos 
la definici\'on $\semBrcks{(\theta_0 \rightarrow \theta'_0) \leq (\theta_1 \rightarrow \theta'_1)}$.
La idea ser\'a ver que la definici\'on que dimos es correcta y adem\'as mostrar de que manera
la podemos construir, comencemos notando que,\\

$\semBrcks{\theta_1 \leq \theta_0} : \semBrcks{\theta_1} \rightarrow \semBrcks{\theta_0} 
= Hom_{\Dom}(\semBrcks{\theta_1},\semBrcks{\theta_0})$\\

$\semBrcks{\theta_0' \leq \theta_1'} : \semBrcks{\theta_0'} \rightarrow \semBrcks{\theta_1'}
= Hom_{\Dom}(\semBrcks{\theta_0'},\semBrcks{\theta_1'})$\\

y definamos entonces dos funtores, uno covariante $Hom(\semBrcks{\theta_1}, \_ )$ y otro contravariante
$Hom(\_,\semBrcks{\theta_1'})$. \\

Tomemos una funci\'on $f$ en $\semBrcks{\theta_0 \rightarrow \theta_0'}$ cualquiera, luego \\

$Hom(\semBrcks{\theta_1}, f)$ $:$ 
$Hom(\semBrcks{\theta_1}, \semBrcks{\theta_0}) \rightarrow Hom(\semBrcks{\theta_1}, \semBrcks{\theta_0'})$,\\

usando lo que notamos al principio podemos hacer, es decir, $\semBrcks{\theta_1 \leq \theta_0} : Hom_{\Dom}(\semBrcks{\theta_1},\semBrcks{\theta_0})$, obtenemos \\

$Hom(\semBrcks{\theta_1}, f) \semBrcks{\theta_1 \leq \theta_0}$ $=$ $f \circ \semBrcks{\theta_1 \leq \theta_0} : 
Hom(\semBrcks{\theta_1}, \semBrcks{\theta_0'})$.\\

Si ahora hacemos algo similar usando el otro funtor tenemos,\\

$Hom(f \circ \semBrcks{\theta_1 \leq \theta_0}, \semBrcks{\theta_1'})$ $:$ 
$Hom(\semBrcks{\theta_0'}, \semBrcks{\theta_1'}) \rightarrow Hom(\semBrcks{\theta_1}, \semBrcks{\theta_1'})$,\\

y aplicando el funtor como antes podemos llegar a la ecuación propuesta,\\

$Hom(f \circ \semBrcks{\theta_1 \leq \theta_0}, \semBrcks{\theta_1'})\semBrcks{\theta_0' \leq \theta_1'}$ $=$
$\semBrcks{\theta_0' \leq \theta_1'} \circ f \circ \semBrcks{\theta_1 \leq \theta_0}$.\\

Luego podemos mencionar que el subtipado para un tipo $\theta \rightarrow \theta'$ es covariante
para $\theta$ y contravariante para $\theta'$.\\

La definici\'on de la relaci\'on $\leq$ entre tipos nos permite ademas actualizar 
la definici\'on de la categor\'ia de contexto, de manera tal de definir $\leq$ entre
contextos para que luego, como pasa con los tipos, esta relaci\'on sea una flecha 
en la categor\'ia. Dados $\pi$ y $\pi'$ tal que $dom \ \pi$ $=$ $dom \ \pi'$, diremos
que $\pi \leq \pi'$ cuando para todo $\iota \in dom \ \pi$ se cumple $\pi \iota \leq \pi' \iota$.\\

\begin{definition}\label{lambdal:contextcategory}
La categor\'ia de contextos, que nombraremos $\Pi$, se define como sigue\\

$\Pi_0$ $=$ $\{\pi \ | \ \pi \ \in \lrangles{Context} \}$\\
\indent
$\Pi_1(\pi,\pi')$ $=$ $\{ \pi \rTo \pi' \ | \ \pi \leq \pi' \}$

\end{definition}

De igual manera que cuando dimos la nueva definici\'on de $\Theta$, podemos dar
una actualizaci\'on a la definici\'on de $\Pi$.

\begin{definition}\label{lambdal:contextsemfunctor}
Sea $\semBrcks{ \ } : \Pi \rightarrow \CD$ un funtor, tal que\\

$\semBrcks{\pi}_0$ $=$ $\prod\limits_{\iota \in dom \ \pi} \semBrcks{\pi\iota}$\\
\indent
$\semBrcks{\pi \leq \pi'}_1$ $=$ $\prod\limits_{\iota \in dom \ \pi} \semBrcks{\pi\iota \leq \pi'\iota}$

\end{definition}

Esta ultima definici\'on da por terminado el trabajo de acomodar los
dominios categ\'oricos, adem\'as de las nuevas formas de jucios de tipado y su 
sem\'antica respectiva.

Para completar la sem\'antica del lenguaje $\lambdaleq$, nos falta 
definir una ecuaci\'on sem\'antica que relacione un juicio de tipado
con una relaci\'on de orden entre dos tipos.\\

\noindent
$\texttt{Denotal Sem:}$ Subsumption.\\

$\semBrcks{\pi \vdash e : \theta'}$ $=$ $\semBrcks{\theta \leq \theta'} \circ \semBrcks{\pi \vdash e : \theta}$

\section{Continuidad y coherencia}

Al igual que como hicimos para el lenguaje $\lambdaarrow$ vamos a probar la
continuidad de las ecuaciones sem\'anticas de $\lambdaleq$. En cuanto a 
esta prueba lo interesante es que es realmente simple, ya que usando la 
prueba de continuidad de $\lambdaarrow$ simplemente nos resta probar 
la continuidad de Subsumption. Por otro lado adem\'as vamos a probar coherencia,
esta propiedad describe que dados dos o mas derivaciones para un mismo
juicio de tipado, todas estas derivaciones tiene el mismo significado
sem\'antico. Coherencia se vuelve interesante para nuestro lenguaje $\lambdaleq$
ya que este tiene subtipado y esto es el que genera que un juicio pueda
tener mas de una derivaci\'on, claramente esto no pasa para $\lambdaarrow$ para
el cual por cada juicio existe una sola derivaci\'on.\\

Empecemos probando la continuidad de las ecuaciones sem\'anticas y luego
introduzcamos mas detalles sobre la coherencia de $\lambdaleq$.

\begin{theorem}

Dado un juicio de tipado $\pi \vdash e : \theta$ la ecuaci\'on sem\'antica
\\ 
$\semBrcks{\pi \vdash e : \theta}$ es una funci\'on continua.

\end{theorem}

\begin{proof}

En la prueba vamos a proceder por inducci\'on en la estructura de la derivaci\'on 
de los juicios de tipado. Adem\'as como ya mencionamos antes, solamente nos resta
probar el caso inductivo para la sem\'antica denotacional de subsumption, supongamos
entonces un juicio de tipado $\pi \vdash e : \theta'$, luego\\

$\semBrcks{\pi \vdash e : \theta'}$ $=$ $\semBrcks{\theta \leq \theta'} \circ \semBrcks{\pi \vdash e : \theta}$\\

por hip\'otesis inductiva obtenemos que $\semBrcks{\pi \vdash e : \theta}$ es una funci\'on
continua, adem\'as por construcci\'on de nuestra categ\'oria de tipos
tenemos que $\semBrcks{\theta \leq \theta'}$ es una funci\'on continua tambi\'en, 
por lo tanto utilizando que la composici\'on de funciones continuas es una funci\'on
continua concluimos que $\semBrcks{\pi \vdash e : \theta'}$ es funci\'on continua.

\end{proof}

Retomemos coherencia con un ejemplo simple considerando el identificador
$\iota$ con tipo $\realt$, para esta frase y tipo existen cuatro derivaciones 
posibles.\\

Usando Subsumption:
\begin{center}
\AxiomC{$\intt \leq \realt$}
\AxiomC{$\iota : \intt \in \pi$}
\UnaryInfC{$\pi \vdash \iota : \intt$}
\BinaryInfC{$\pi \vdash \iota : \realt$}
\DisplayProof
\end{center}

\

Usando Subsumption$'$:
\begin{center}
\AxiomC{$\intt \leq \realt$}
\AxiomC{$\boolt \leq \intt$}
\AxiomC{$\iota : \boolt \in \pi'$}
\UnaryInfC{$\pi' \vdash \iota : \boolt$}
\BinaryInfC{$\pi' \vdash \iota : \intt$}
\BinaryInfC{$\pi' \vdash \iota : \realt$}
\DisplayProof
\end{center}

\

\

Usando Subsumption$''$:
\begin{center}
\AxiomC{$\boolt \leq \intt$}
\AxiomC{$\intt \leq \realt$}
\BinaryInfC{$\boolt \leq \realt$}
\AxiomC{$\iota : \boolt \in \pi''$}
\UnaryInfC{$\pi'' \vdash \iota : \boolt$}
\BinaryInfC{$\pi'' \vdash \iota : \realt$}
\DisplayProof
\end{center}

\

\

Usando la regla del identificador:
\begin{center}
\AxiomC{$\iota : \realt \in \pi'''$}
\UnaryInfC{$\pi''' \vdash \iota : \realt$}
\DisplayProof
\end{center}

\

Ahora bien, algo importante a notar es que mencion\'abamos que coherencia 
enunciaba la igualdad sem\'antica de mismos juicios de tipado y en nuestro 
ejemplo estamos considerando juicios de tipado que no son iguales debido 
a los diferentes contextos. Esto se debe por un lado a que tiene sentido
hablar de coherencia de juicios de tipado cuando consideramos juicios 
para frases cerradas ya que en ese caso siempre consideramos el contexto 
vac\'io, pero por otro lado coherencia en general enuncia la igualdad 
sem\'antica de derivaciones de un juicio de tipado para una frases con 
determinado tipo, independientemente del contexto.\\

Otra cosa que surge de analizar este ejemplo es como saber cuantas derivaciones
posibles existen para una determinada frase y tipo, acabamos de ver que
incluso para el identificador con un tipo simple existen muchas opciones.
Recordando que nos interesa seriamente saber la cantidad de derivaciones 
que pueda tener una frase para poder asegurar que todas tienen exactamente
el mismo significado sem\'antico. Para hacer frente a esto es que usamos 
el lema de inversi\'on, pero antes de enunciarlo hagamos una ultima observaci\'on.
Adelantando que vamos a utilizar inducci\'on estructural en la derivaci\'on
del juicio de tipado para probar coherencia, notar que podemos agrupar
tres de las derivaciones en un solo paso que es aplicar 
subsumption como ultima regla para obtener el juicio de tipado final. En
conclusi\'on nuestro lema de inversi\'on nos dir\'a que dado un juicio de 
tipado cualquiera la ultima regla aplicada para concluirlo fue, o bien
la regla respectiva a su frase o bien subsumption.

\begin{lemma}[De inversi\'on]
Sea $\pi \vdash e : \theta$ un juicio de tipado valido cualquiera, entonces
sucedi\'o alguna de la siguientes:

\begin{itemize}
\item Si $e$ es constante, la ultima regla aplicada fue (Constantes) o (Subsumption)
\item Si $e$ es identificador, la ultima regla aplicada fue (Identificador) o (Subsumption)
\item Si $e = \odot e'$, la ultima regla aplicada fue ($\odot$) o (Subsumption)
\item Si $e = e' \circledcirc e''$, la ultima regla aplicada fue 
($\circledcirc$) o (Subsumption)
\item Si $e = \cifthenelse{b}{e'}{e''}$, la ultima regla aplicada fue 
(Expresi\'on condicional) o (Subsumption)
\item Si $e = e'e''$, la ultima regla aplicada fue (Aplicaci\'on) o (Subsumption)
\item Si $e = \clambda{\iota}{\theta}{e'}$, la ultima regla aplicada fue 
(Abstracci\'on lambda) o (Subsumption)
\item Si $e = \rec{e'}$, la ultima regla aplicada fue (Operador de punto fijo) o (Subsumption)
\end{itemize}

\end{lemma}

Antes de enunciar y comenzar con la prueba de coherencia vamos a definir informalmente una 
funci\'on que ser\'a de utilidad para agilizar la prueba y definir algunas
propiedades. Esta funci\'on estricta la nombraremos $\J_{\theta}^{\theta'}$ tal que
va de $\semBrcks{\theta}$ en $\semBrcks{\theta'}$ y los posibles casos se defininen as\'i\\

$\J_{\boolt}^{\intt} = \lambda b .$ if $b$ then $0$ else $1$\\
\indent

$\J_{\intt}^{\realt} = \J$ (la inyecci\'on en los reales)\\
\indent

$\J_{\boolt}^{\realt} = \J_{\intt}^{\realt} \circ \J_{\boolt}^{\intt}$\\

\begin{proposition}
Sea $\J_{\theta}^{\theta'}$, dados $z$, $z'$ en $\semBrcks{\theta}$, dos operadores
unarios $\odot : \semBrcks{\theta} \rightarrow \semBrcks{\theta}$ y 
$\odot' : \semBrcks{\theta'} \rightarrow \semBrcks{\theta'}$ y dos operadores
binarios $\circledcirc : \semBrcks{\theta} \rightarrow 
		\semBrcks{\theta} \rightarrow \semBrcks{\theta}$ y
$\circledcirc' : \semBrcks{\theta'} \rightarrow 
		\semBrcks{\theta'} \rightarrow \semBrcks{\theta'}$
cualesquiera entonces vale

\begin{itemize}
\item $\J_{\theta}^{\theta'} (\iotabot z) = \iotabot (\J_{\theta}^{\theta'} z)$
\item $\J_{\theta}^{\theta'} (\iotabot (\odot z)) = \iotabot (\odot' (\J_{\theta}^{\theta'} z))$
\item $\J_{\theta}^{\theta'} (\iotabot (z \circledcirc z')) = 
	   \iotabot ((\J_{\theta}^{\theta'} z) \circledcirc' (\J_{\theta}^{\theta'} z'))$

\end{itemize}

\end{proposition}


\begin{theorem}[De coherencia]
Sean $e$ una frase y $\theta$ un tipo cualesquiera y tal que tenemos 
$n$ derivaciones de los juicios de tipado
$\pi_1 \vdash e : \theta, \ldots, \pi_n \vdash e : \theta$, donde
$\pi_i \leq \pi_j$ o $\pi_j \leq \pi_i$ para $i,j = 1 \ldots n$ y
$\eta_i : \semBrcks{\pi_i}$ para $i = 1 \ldots n$, entonces

\begin{center}
$\semBrcks{\pi_1 \vdash e : \theta}\eta_1 = \ldots = \semBrcks{\pi_n \vdash e : \theta}\eta_n$
\end{center}

\end{theorem}

\begin{proof}
En la prueba vamos a proceder por inducci\'on es la estructura de la derivaci\'on 
de los juicios de tipado, como ya mencionamos antes, la idea es usar inversi\'on.\\

\begin{itemize}

\item Casos base
\begin{itemize}
\item Supongamos $e$ es constante y tomemos un $\eta : \semBrcks{\pi}$ cualquiera, 
luego por inversi\'on tenemos\\

Usando (Constante)\\

$\semBrcks{\pi \vdash e : \theta}\eta = \iotabot e$\\

Usando (Subsumption)\\

$\semBrcks{\pi \vdash e : \theta}\eta = 
\semBrcks{\theta' \leq \theta}(\semBrcks{\pi \vdash e : \theta'}\eta)$ $=$
$\J_{\theta'}^{\theta} (\iotabot e)$ $=$ $\iotabot (\J_{\theta'}^{\theta} e)$\\

Empecemos notando que por la definici\'on que hicimos de la funci\'on $\J_{\theta}^{\theta'}$
podemos suponer sin que se nos escape ning\'un caso de que cuando usamos primero
subsumption, despu\'es $\semBrcks{\pi \vdash e : \theta} = \iotabot e$. Por otro lado
restar\'ia probar para este caso que $e : \semBrcks{\theta}$  es igual a 
$\J_{\theta'}^{\theta} e : \semBrcks{\theta}$, pero esto es directo de la definici\'on
de la funci\'on $\J_{\theta'}^{\theta}$.

\item Supongamos $e = \iota$ y tomemos $\eta : \semBrcks{\pi}$, $\eta' : \semBrcks{\pi'}$
cualesquiera, notar que estamos suponiendo $eta$'s y $\pi$'s distintos ya que 
no puede existe en un mismo contexto un identificador con dos tipos distintos, como 
antes usando inversi\'on tenemos\\

Usando (Identificador)\\

$\semBrcks{\pi \vdash \iota : \theta}\eta = \eta \iota$\\

Usando (Subsumption)\\

$\semBrcks{\pi' \vdash \iota : \theta}\eta' = 
\semBrcks{\theta' \leq \theta} (\semBrcks{\pi' \vdash \iota : \theta'}\eta') = 
\J_{\theta'}^{\theta} (\eta' \iota)$\\

Como paso para el caso de las constantes, restar\'ia ver que 
$\eta \iota = \J_{\theta'}^{\theta} (\eta' \iota)$, pero de nuevo esto es directo
de suponer $\eta \iota = \iotabot z$ y $\eta' \iota = \iotabot z'$, con
$z:\theta$ y $z':\theta'$.

\end{itemize}

\item Casos inductivos

Supongamos un $e : \theta$ tal que vale
$\semBrcks{\pi_1 \vdash e : \theta}\eta_1 = \ldots = \semBrcks{\pi_n \vdash e : \theta}\eta_n$ 
y probemos
que para un nuevo $e : \theta$ de mayor complejidad vale\\
$\semBrcks{\pi_1 \vdash e : \theta}\eta_1 = \ldots = 
\semBrcks{\pi_n \vdash e : \theta}\eta_n$. \\

Las pruebas de los distintos casos de inducci\'on van a proceder de la siguiente manera,
vamos a partir de suponer la ultima regla fue la del comando en si y probar que llegamos
a que la ultima regla fue subsumption o viceversa, adem\'as vamos a asumir que nuestros
programas siempre terminan, para los casos en los que no es directo demostrar la igualdad.

\begin{itemize}
\item Supongamos $e = \odot e'$, luego\\

$\semBrcks{\pi \vdash \odot e' : \theta}\eta$ $=$ 
$(\lambda x . \ \iotabot (\odot x))_{\dbot} (\semBrcks{\pi \vdash e' : \theta}\eta)$\\

utilizar hip\'otesis inductiva ac\'a significar\'ia que podemos suponer que la
ultima regla para $\pi \vdash e' : \theta$ fue cualquiera de las posibilidades que
da inversi\'on entonces\\

$(\lambda x . \ \iotabot (\odot x))_{\dbot} (\semBrcks{\pi \vdash e' : \theta}\eta)$ $=$ \\
$(\lambda x . \ \iotabot (\odot x))_{\dbot} 
	(\semBrcks{\theta' \leq \theta}(\semBrcks{\pi \vdash e' : \theta'}\eta))$ $=$ \\
$\iotabot (\odot (\J_{\theta'}^{\theta} \semBrcks{\pi \vdash e' : \theta'}\eta)))$\\

usando la propiedad sobre la inyecci\'on $\J_{\theta'}^{\theta}$ tenemos\\

$\iotabot (\odot (\J_{\theta'}^{\theta} \semBrcks{\pi \vdash e' : \theta'}\eta)))$ $=$
$\J_{\theta'}^{\theta} (\iotabot (\odot \semBrcks{\pi \vdash e' : \theta'}\eta))$ $=$\\
$\semBrcks{\theta' \leq \theta}(\semBrcks{\pi \vdash \odot e' : \theta'}\eta)$ $=$
$\semBrcks{\pi \vdash \odot e' : \theta}\eta$\\

Y con esto hemos finalizado la prueba de este caso, repasando, partimos de suponer
la ultima regla usada fue $\odot$ y llegamos a que la ultima regla usada fue subsumption.

\item (An\'alogo para el caso de operadores binarios)

\item Supongamos $e = \cifthenelse{b}{e'}{e''}$ y 
$\semBrcks{\pi \vdash b : \boolt}\eta = \iotabot \ true$\\

$\semBrcks{\pi \vdash \cifthenelse{b}{e'}{e''} : \theta}\eta$ $=$\\
\indent \ \ \ \ \ \ \ \ \ \ \ \ \ \ \ \ \ \ \ \
$(\lambda b. \ if \ b \ then \ \semBrcks{\pi \vdash e' : \theta}\eta$\\
\indent \ \ \ \ \ \ \ \ \ \ \ \ \ \ \ \ \ \ \ \ \ \ \ \ \ \ \ \ \ \ \ \ \ \ \ \
$else \ \semBrcks{\pi \vdash e'' : \theta}\eta)_{\dbot} 
							(\semBrcks{\pi \vdash b : \boolt}\eta)$ $=$\\
$\semBrcks{\pi \vdash e' : \theta}\eta$

usando hip\'otesis inductiva como en el caso del operador unario, tenemos que\\

$\semBrcks{\pi \vdash e' : \theta}\eta$ $=$ 
$\semBrcks{\theta' \leq \theta}(\semBrcks{\pi \vdash e' : \theta'}\eta)$ $=$\\
$\semBrcks{\theta' \leq \theta}($
$(\lambda b. \ if \ b \ then \ \semBrcks{\pi \vdash e' : \theta'}\eta$\\
\indent \ \ \ \ \ \ \ \ \ \ \ \ \ \ \ \ \ \ \ \ \ \ \ \ \ \ \ \ \ \ \ \
$else \ \semBrcks{\pi \vdash e'' : \theta'}\eta)_{\dbot} 
						(\semBrcks{\pi \vdash b : \boolt}\eta))$ $=$\\
$\semBrcks{\theta' \leq \theta}
	(\semBrcks{\pi \vdash \cifthenelse{b}{e'}{e''} : \theta'}\eta)$ $=$\\
$\semBrcks{\pi \vdash \cifthenelse{b}{e'}{e''} : \theta}\eta$\\

An\'alogo si suponemos $\semBrcks{\pi \vdash b : \boolt}\eta = \iotabot \ false$.

\item Supongamos $e = e'e''$, luego\\

$\semBrcks{\pi \vdash e'e'' : \theta}\eta$ $=$ 
$\semBrcks{\pi \vdash e' : \thetahat \rightarrow \theta}\eta
							(\semBrcks{\pi \vdash e'' : \thetahat}\eta)$ $=$\\
$(\semBrcks{\thetahat' \rightarrow \theta' \leq \thetahat \rightarrow \theta}
(\semBrcks{\pi \vdash e' : \thetahat' \rightarrow \theta'}\eta))$\\
\indent \ \ \ \ \ \ \ \ \ \ \ \ \ \ \ \ \ \ \ \ \ \ \ \ \ \ \ \ \ \ \ \ \ \ \ \ \ \ \ \ \
$(\semBrcks{\thetahat' \leq \thetahat}(\semBrcks{\pi \vdash e'' : \thetahat'}\eta))$ $=$\\
$(\semBrcks{\theta' \leq \theta} 
	\circ 
\semBrcks{\pi \vdash e' : \thetahat' \rightarrow \theta'}\eta)
	\circ
 \semBrcks{\thetahat \leq \thetahat'})$\\
\indent \ \ \ \ \ \ \ \ \ \ \ \ \ \ \ \ \ \ \ \ \ \ \ \ \ \ \ \ \ \ \ \ \ \ \ \ \ \ \ \ \
$(\semBrcks{\thetahat' \leq \thetahat}(\semBrcks{\pi \vdash e'' : \thetahat'}\eta))$\\

hasta ac\'a hemos usado dos veces hip\'otesis inductiva y la definici\'on
de la ecuaci\'on sem\'antica 
$(\semBrcks{\thetahat' \rightarrow \theta' \leq \thetahat \rightarrow \theta}$,
notar que 
$\semBrcks{\thetahat \leq \thetahat'} \circ \semBrcks{\thetahat' \leq \thetahat}$
$=$ $\semBrcks{\thetahat' \leq \thetahat'}$ $=$ $1_{\semBrcks{\thetahat'}}$\\

$(\semBrcks{\theta' \leq \theta} 
	\circ 
\semBrcks{\pi \vdash e' : \thetahat' \rightarrow \theta'}\eta)
	\circ
 \semBrcks{\thetahat \leq \thetahat'})$\\
\indent \ \ \ \ \ \ \ \ \ \ \ \ \ \ \ \ \ \ \ \ \ \ \ \ \ \ \ \ \ \ \ \ \ \ \ \ \ \ \ \ \
\ \ \ \ \ \ \ \ \ \ \ \
$(\semBrcks{\thetahat' \leq \thetahat}(\semBrcks{\pi \vdash e'' : \thetahat'}\eta))$ $=$\\
$(\semBrcks{\theta' \leq \theta} 
	\circ 
\semBrcks{\pi \vdash e' : \thetahat' \rightarrow \theta'}\eta)
	\circ
\semBrcks{\thetahat \leq \thetahat'} 
 	\circ
\semBrcks{\thetahat' \leq \thetahat})$ \\
\indent \ \ \ \ \ \ \ \ \ \ \ \ \ \ \ \ \ \ \ \ \ \ \ \ \ \ \ \ \ \ \ \ \ \ \ \ \ \ \ \ \
\ \ \ \ \ \ \ \ \ \ \ \ \ \ \ \ \ \ \ \ \ \ \ \ \ \ \ \ \ \
$(\semBrcks{\pi \vdash e'' : \thetahat'}\eta)$ $=$\\
$\semBrcks{\theta' \leq \theta} 
(\semBrcks{\pi \vdash e' : \thetahat' \rightarrow \theta'}\eta 
(\semBrcks{\pi \vdash e'' : \thetahat'}\eta))$ $=$\\
$\semBrcks{\theta' \leq \theta}(\semBrcks{\pi \vdash e'e'' : \theta'}\eta)$ $=$\\
$\semBrcks{\pi \vdash e'e'' : \theta}\eta$

\item Supongamos $e = \clambda{\iota}{\theta}{e'}$ y un $z : \semBrcks{\theta}$ cualquiera, luego\\

$\semBrcks{\pi \vdash \clambda{\iota}{\theta}{e'} : \theta \rightarrow \theta'}\eta \ z$ $=$\\
$(\semBrcks{\thetahat \rightarrow \thetahat' \leq \theta \rightarrow \theta'}
(\semBrcks{\pi \vdash \clambda{\iota}{\thetahat}{e'} : \thetahat \rightarrow \thetahat'}\eta))z$
$=$\\
$(\semBrcks{\thetahat' \leq \theta'} 
	\circ
(\semBrcks{\pi,\iota:\thetahat \vdash e' : \thetahat'} \circ (ext_{\iota,\thetahat}\eta))
	\circ
\semBrcks{\theta \leq \thetahat}) z$ $=$\\
$\semBrcks{\pi,\iota:\thetahat \vdash e' : \theta'}
		(ext_{\iota,\thetahat} \ \eta \ (\semBrcks{\theta \leq \thetahat}z))$ $=$\\
$\semBrcks{\pi,\iota:\thetahat \vdash e' : \theta'}
		[ \ \eta \ | \ \iota:\semBrcks{\theta \leq \thetahat}z \ ]$\\

Esta vez empezamos suponiendo que la ultima regla fue subsumption, desarrollamos
utilizando la definici\'on de la ecuaci\'on para el subtipado del tipo flecha y
aplicamos la definici\'on, para el otro lado, de subsumption para el cuerpo de la 
abstracci\'on lambda. Todo esto deja listo para aplicar hip\'otesis inductiva\\

$\semBrcks{\pi,\iota:\thetahat \vdash e' : \theta'}
		[ \ \eta \ | \ \iota:\semBrcks{\theta \leq \thetahat}z \ ]$ $=$\\
$\semBrcks{\pi,\iota:\theta \vdash e' : \theta'} [ \ \eta \ | \ \iota:z \ ]$ $=$\\
$(\semBrcks{\pi,\iota:\theta \vdash e' : \theta'} \circ (ext_{\iota,\theta} \eta)) z$ $=$\\
$\semBrcks{\pi \vdash \clambda{\iota}{\theta}{e'} : \theta \rightarrow \theta'}\eta \ z$\\

Lo que nos restar\'ia probar entonces es que $\pi,\iota:\theta \leq \pi,\iota:\thetahat$,
pero esto es directo, ya que supusimos $\theta \leq \thetahat$.

\item Supongamos $e = \rec{e'}$, luego \\

$\semBrcks{\pi \vdash \rec{e'} : \theta} \eta$ $=$ 
$\semBrcks{\theta' \leq \theta} (\semBrcks{\pi \vdash \rec{e'} : \theta'} \eta)$ $=$\\
$\semBrcks{\theta' \leq \theta} 
	(\Y_{\theta} (\semBrcks{\pi \vdash e' : \theta' \rightarrow \theta'}\eta))$ $=$\\
$\semBrcks{\theta' \leq \theta} 
	(\bigsqcup\limits^{\infty}_{i=0} 
	(\semBrcks{\pi \vdash e' : \theta' \rightarrow \theta'}\eta)^{i} \bot_{\theta'})$ $=$\\
$\bigsqcup\limits^{\infty}_{i=0} 
	((\semBrcks{\theta' \leq \theta} \circ
	(\semBrcks{\pi \vdash e' : \theta' \rightarrow \theta'}\eta)^{i}) \bot_{\theta'})$ $=$\\
$\bigsqcup\limits^{\infty}_{i=0} 
	((\semBrcks{\theta' \leq \theta} 
	\circ
	(\semBrcks{\pi \vdash e' : \theta' \rightarrow \theta'}\eta)^{i}
	\circ
	\semBrcks{\theta \leq \theta'} 
	) \bot_{\theta})$ $=$\\
$\bigsqcup\limits^{\infty}_{i=0} 
	((\semBrcks{\theta' \rightarrow \theta' \leq \theta \rightarrow \theta} 
	 (\semBrcks{\pi \vdash e' : \theta' \rightarrow \theta'}\eta)^{i}
	) \bot_{\theta})$ $=$\\
$\bigsqcup\limits^{\infty}_{i=0} (
	(\semBrcks{\pi \vdash e' : \theta \rightarrow \theta}\eta)^{i}
	\bot_{\theta})$ $=$ 
$\Y_{\theta} (\semBrcks{\pi \vdash e' : \theta \rightarrow \theta}\eta)$ $=$\\
$\semBrcks{\pi \vdash \rec{e'} : \theta}\eta$

\end{itemize}

\end{itemize}

\end{proof}

\section{Implementaci\'on en Idris}

Sintaxis del subtipado
\begin{code}
data (<~) : PhraseType -> PhraseType -> Type where
    IntExpToRealExp : IntExp  <~ RealExp
    BoolExpToIntExp : BoolExp <~ IntExp
    
    Reflx : (t:PhraseType) -> t <~ t
    Trans : {t:PhraseType} -> {t':PhraseType} -> {t'':PhraseType} -> 
            t <~ t' -> t' <~ t'' -> t <~ t''
            
    SubsFun : {t0:PhraseType} -> {t0':PhraseType} -> 
              {t1:PhraseType} -> {t1':PhraseType} -> 
              t0 <~ t0' -> t1 <~ t1' -> (t0' :-> t1) <~ (t0 :-> t1')
\end{code}

\noindent Sem\'antica para el subtipado
\begin{code}
evalLeq : {t:PType} -> {t':PType} -> t <~ t' -> evalTy t -> evalTy t'
evalLeq IntExpToRealExp    = prim__intToFloat
evalLeq BoolExpToIntExp    = prim__boolToInt
evalLeq {t'=t} (Reflx t)   = id
evalLeq (Trans leq leq')   = evalLeq leq' . evalLeq leq
evalLeq (SubsFun leq leq') = \f => evalLeq leq' . f . evalLeq leq
\end{code}

\noindent Sem\'antica para el lenguaje $\lambdaleq$
\begin{code}
using (Pi:Ctx, Theta:PType, Theta':PType)
    data TypeJugdmnt : Ctx -> PhraseType -> Type where
        I     : (i:Identifier) -> InCtx Pi i -> TypeJugdmnt Pi Theta
        CInt  : Int   -> TypeJugdmnt Pi IntExp
        CBool : Bool  -> TypeJugdmnt Pi BoolExp
        CReal : Float -> TypeJugdmnt Pi RealExp
        
        Lam   : (i:Identifier) -> (pt:PhraseType) -> (fi:Fresh Pi i) ->
                TypeJugdmnt (Prepend Pi i pt fi) Theta' -> 
                TypeJugdmnt Pi (pt :-> Theta')
        App   : TypeJugdmnt Pi (Theta :-> Theta') -> TypeJugdmnt Pi Theta -> 
                TypeJugdmnt Pi Theta'
        Rec   : TypeJugdmnt Pi (Theta :-> Theta) -> TypeJugdmnt Pi Theta
        
        If    : TypeJugdmnt Pi BoolExp -> TypeJugdmnt Pi Theta -> 
                TypeJugdmnt Pi Theta -> TypeJugdmnt Pi Theta
        BinOp : (evalTy a -> evalTy b -> evalTy c) -> 
                TypeJugdmnt Pi a -> TypeJugdmnt Pi b -> TypeJugdmnt Pi c
        UnOp  : (evalTy a -> evalTy b) -> TypeJugdmnt Pi a -> TypeJugdmnt Pi b
        
        Subs    : Theta <~ Theta' -> TypeJugdmnt Pi Theta -> TypeJugdmnt Pi Theta'

eval : {Pi:Ctx} -> {Theta:PType} -> TypeJugdmnt Pi Theta -> evalCtx Pi -> evalTy Theta
eval (Subs leq p) eta = evalLeq leq $ eval p eta
eval {Pi=p} {Theta=pt} (I i iIn) eta = search p i pt iIn eta
eval (CInt x)    eta = x
eval (CBool x)    eta = x
eval (CReal x)    eta = x
eval {Pi=p} (Lam i pt fi b) eta = \z => eval b (update p eta i pt z fi)
eval (App e e')   eta = (eval e eta) (eval e' eta)
eval (Rec e) eta = fix (eval e eta)
eval (If b e e')  eta = if eval b eta then eval e eta else eval e' eta
eval (BinOp op x y) eta = op (eval x eta) (eval y eta)
eval (UnOp op x) eta = op (eval x eta)

\end{code}
