\chapter{Conclusi\'on}

La tesis ten\'ia como objetivo definir las ecuaciones sem\'anticas
y dar una implementaci\'on del lenguaje Forsythe. El plan entonces fue
comenzar con un lenguaje simple e ir complejizando su definici\'on hasta
llegar a Forsythe; empezamos con $\lambdaarrow$ que es el c\'alculo lambda
simplemente tipado, luego agreg\'abamos subtipado en $\lambdaleq$, el siguiente
paso fue agregar aspectos imperativos y obtener $\Alike$, este ya 
cerca de Forsythe. Y una vez con $\Alike$ definido dar el salto hacia 
Forsythe.\\

Lo que se realizo en la tesis fue la definici\'on e implementaci\'on de los 
lenguajes $\lambdaarrow$, $\lambdaleq$ y $\Alike$. Para cada uno adem\'as 
probamos distintas propiedades como continuidad, naturalidad o coherencia 
de la ecuaciones sem\'anticas.\\

En cuanto a $\lambdaarrow$ y $\lambdaleq$ podemos mencionar que fueron
especialmente pr\'acticos al momento de estudiar los sistemas de tipado y
sobre todo para comenzar con la implementaci\'on en un lenguaje de programaci\'on
con tipos dependientes. Adem\'as desde el punto de vista de las propiedades
fue muy interesante pensar el significado de coherencia para $\lambdaleq$ y
su contraste con \cite[Prop 4]{coherencereynolds}.\\

Para $\Alike$ surgi\'o la idea de dar una sem\'antica que no contemple stack
discipline, adem\'as de la sem\'antica general con stack discipline que caracteriza 
a este tipo de lenguajes. La sem\'antica entonces que se defini\'o intento ser
lo mas general posible posible utilizando categor\'ias
de manera de poder, cambiando lo m\'inimo posible, seleccionar entre una
sem\'antica u otra. El secreto parec\'ia estar en tener una categor\'ia
de estados con algunas pequeñas restricciones y luego tener dos versiones
de esta categor\'ia.\\
La implementaci\'on de $\Alike$ termino difiriendo un poco de la sem\'antica
denotacional que dimos y de la cual probamos ciertas propiedades, esto 
debido principalmente por la utilizaci\'on de categor\'ias. En contraste
con las implementaciones de $\lambdaarrow$ y $\lambdaleq$ que respetan 
bien la estructura de la sem\'antica denotacional. En particular la definici\'on
de la categor\'ia de estados que hac\'ia falta para la sem\'antica denotacional
no aparece en la implementaci\'on sino que se utilizo una idea an\'aloga, no
categ\'orica, que se puede encontrar en \cite[Cap 19]{reynolds2009theories}.\\

Las cosas que quedaron por hacer y pueden ser parte de trabajos futuros,
fueron definir e implementar 
la sem\'antica de Forsythe, para lo cual se propone definir un lenguaje 
intermedio entre $\Alike$ y Forsythe, $\lambdainter$. El cual es b\'asicamente
$\Alike$ pero tal que el sistema de tipos soporta intersecci\'on, una prueba
muy interesante para $\lambdainter$ ser\'ia coherencia.\\
Adem\'as por el lado de la implementaci\'on ser\'ia bueno implementar categor\'ias
para despu\'es dar una implementaci\'on que se corresponda con la sem\'antica
denotacional que se dio para $\Alike$. Luego dar una implementaci\'on de 
$\lambdainter$ parece ser bastante directo en base a la de $\Alike$.
