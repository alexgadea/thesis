\chapter{Introducci\'on}
\label{chap:intro}

Como lo sugiere el titulo de la tesis, esta tratara sobre el estudio
de las sem\'anticas para el lenguaje Forsythe propuesto por John C. Reynolds
(ref ac\'a). B\'asicamente cuando hablamos de estudiar las sem\'anticas del 
lenguaje a lo que nos referimos es, dada la sintaxis definida por Reynolds
queremos interpretar las palabras que se forman en base a esta. Para interpretar
las palabras vamos necesitar un modelo matem\'atico de manera de que
cada palabra del lenguaje se corresponda con un elemento de este modelo, 
en general vamos a hacer uso de dos modelos, el categ\'orico y el conjuntista.\\

Para llegar a definir la sem\'antica de Forsythe vamos a definir tres lenguajes
mas simples, empezando con un lenguaje funcional con tipado simple el cual 
llamaremos Lambda arrow, luego extenderemos \'unicamente el sistema de tipos para 
que soporte ahora subtipado, este se llamara Lambda leq y finalmente extenderemos 
el lenguaje y los tipos con el fin de obtener un lenguaje funcional e imperativo 
cuyo nombre sera Algol-like, este ultimo ya bastante cercano a Forsythe. 

Por otro lado ademas de las definiciones teoricas de los lenguajes, vamos a 
implementar los evaluadores correspondientes para cada uno, para esto nos va a 
hacer falta usar un lenguaje de programación con soporte para tipos dependientes.

El segundo cap\'itulo de la tesis va a incluir la definici\'on de los lenguajes
lambda arrow y lambda leq, esto ser\'a presentar la sint\'axis y las ecuaciones
sem\'anticas, ademas vamos a introducir todos los conceptos relacionados al 
sistemas de tipo simple y con subtipado.

El tercer cap\'itulo tratar\'a sobre el lenguaje Algol-like, como en el segundo
cap\'itulo, vamos a presentar la sint\'axis y las ecuaciones sem\'anticas, ademas
con la base de tipos del cap\'itulo anterior vamos a actualizar la sint\'axis y
sem\'antica de estos tipos.

¿El cuarto trata de Forsythe?\\

La tesis requerir\'a conocimientos previos sobre teor\'ia de dominios 
(predominios, dominios y funciones continuas), teor\'ia
de categor\'ias (categor\'ias concretas y funtoriales, producto, 
objeto exponencial, funtores y pullback) y sem\'antica denotacional. Un
conocimiento extra sobre tipos dependientes resultara practico exclusivamente
para la parte de implementaci\'on.
