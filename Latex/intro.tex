\chapter{Introducci\'on}
\label{chap:intro}

Como lo sugiere el titulo de la tesis, esta tratara sobre el estudio
de sem\'anticas categ\'oricas para lenguajes Algol-like. Para esto
vamos a estudiar tres lenguajes, $\lambdaarrow$, $\lambdaleq$ y $\Alike$,
este ultimo perteneciente a la clase de lenguajes Algol-like y para el cual
vamos a dar dos definiciones sem\'anticas.
La base de los lenguajes que vamos a estudiar se puede encontrar en los
cap\'itulos 10 y 19 de (ref a reynolds TheoriesOf...), la idea de estudiar
categ\'oricamente estos es en realidad motivado por la doble definici\'on
sem\'antica que pretendemos dar de $\Alike$. Vamos a querer mostrar, 
en alg\'un sentido se hizo, que podemos cambiar el significado
sem\'antico de un lenguaje sin cambiar sus ecuaciones, simplemente
cambiando ciertas categor\'ias. Adem\'as algo interesante de dar
la sem\'antica utilizando categor\'ias ser\'a en las pruebas que demos 
de cada lenguaje en el sentido de propiedades que utilizamos sobre los
dominios que de otra manera habr\'ia que probar por otro lado.\\

El trabajo empezar\'a definiendo el lenguaje $\lambdaarrow$, este ser\'a
un lenguaje funcional con tipado simple que mas allá
de su sem\'antica conocida sirve como punto de partida para la introducci\'on
de conceptos generales sobre los lenguajes que vamos a estudiar, como pueden
ser los contextos, la sem\'antica intr\'inseca para los tipos o la definici\'on
de sem\'antica para un juicio de tipado como manera de representar el significado
de las frases de nuestro lenguaje, entre otras cuestiones relacionas. Adem\'as
la idea es presentar una implementaci\'on, para cada lenguaje, en un 
lenguaje de programaci\'on con tipos dependientes, en particular para esta 
tarea ser\'a bastante \'util empezar considerando este lenguaje simple.\\

El siguiente paso va a ser tomar a $\lambdaarrow$ y extender su sistema de 
tipos para que incluya subtipado, a este nuevo lenguaje lo llamaremos
$\lambdaleq$ el cual entonces seguir\'a siendo un lenguaje funcional
pero ahora con un sistema de tipos con subtipado. Algo interesante de
esta extensi\'on del lenguaje $\lambdaarrow$, que nos define $\lambdaleq$,
es que vamos a reutilizar completamente todas las definiciones que hicimos
en $\lambdaarrow$. Siguiendo con esta idea adem\'as distintos conceptos sobre 
el subtipado que introduzcamos en $\lambdaleq$ van a ser generales para
cualquier lenguaje con subtipos, en particular para nuestro lenguaje final
$\Alike$.

Por otro lado, adem\'as de las definiciones, vamos a dar algunas pruebas
interesantes sobre propiedades de estos lenguajes, algo bueno, y que tiene
que ver con lo que mencion\'abamos antes, es que las pruebas que realicemos
para $\lambdaarrow$ van a ser, o totalmente validas para $\lambdaleq$, o
van a servir para completar pasos en las pruebas de $\lambdaleq$.\\

Para finalizar vamos a extender el lenguaje $\lambdaleq$, ac\'a extender
ya no ser\'a tan simple y directo como paso de $\lambdaarrow$ a $\lambdaleq$
y habr\'a casi que empezar desde cero, mas si tenemos en cuenta que vamos a
dar dos definiciones sem\'anticas distintas, a este lenguaje entonces lo 
llamaremos $\Alike$. Este ser\'a un lenguaje Algol-like el cual
combina aspectos imperativos y funcionales, este tipo de lenguajes adem\'as
consideran stack discipline, que ser\'a la raz\'on por la cual vamos a
definir dos sem\'anticas, una que lo considere y otra
que no, lo interesante ser\'a poder lograrlo cambiando la menor
cantidad de definiciones posibles. Adem\'as, como sucedi\'o para nuestros
lenguajes anteriores, vamos a probar algunas propiedades de $\Alike$.\\

El segundo cap\'itulo, puntualmente va a incluir la definici\'on de los lenguajes
$\lambdaarrow$ y $\lambdaleq$, esto ser\'a presentar la sint\'axis y las ecuaciones
sem\'anticas, ademas vamos a introducir todos los conceptos relacionados al 
sistemas de tipo simple y con subtipado, para finalizar vamos a dar pruebas de
continuidad de las ecuaciones sem\'anticas y vamos a probar la correcci\'on
de $\lambdaarrow$ con respecto a la regla-$\beta$.\\

El tercer cap\'itulo tratar\'a sobre el lenguaje $\Alike$, como en el segundo
cap\'itulo, vamos a presentar la sint\'axis y las ecuaciones sem\'anticas. Para
este lenguaje, como ya mencionamos, vamos a dar dos sem\'anticas distintas, adem\'as
vamos a dar una prueba de la naturalidad de las ecuaciones sem\'anticas.\\

La tesis requerir\'a conocimientos previos sobre teor\'ia de dominios 
(predominios, dominios y funciones continuas), teor\'ia
de categor\'ias (categor\'ias concretas y funtoriales, producto, 
objeto exponencial, funtores y pullback) y sem\'antica denotacional. Un
conocimiento extra sobre tipos dependientes resultara practico exclusivamente
para la parte de implementaci\'on.
