\chapter{Introducci\'on}
\label{chap:intro}

Como lo sugiere el titulo de la tesis, esta tratara sobre el estudio
de las sem\'anticas para el lenguaje Forsythe propuesto por John C. Reynolds
(ref ac\'a). B\'asicamente cuando hablamos de estudiar las sem\'anticas del 
lenguaje a lo que nos referimos es, dada la sintaxis definida por Reynolds
queremos intepretar las frases o palabras que se forman en base a esta.\

En las siguientes secciones vamos a introducir algunos conceptos b\'asicos
sobre el lenguaje y relacionados.

\section{Forsythe}
Est\'e es un lenguaje Algol-like \'el cual Reynolds pretende, segun menciona 
en (ref ac\'a), sea lo mas uniforme y general posible. El lenguaje es tipado,
incluyendo subtipos e intersección, su paradigma ademas es imperativo y 
funcional y forma de evaluaci\'on normal.

\section{Tipado}
Se podr\'ia decir que programar en un cierto lenguaje termina estando intimamente
relacionado con quitar comportamientos no deseados, esta tarea no siempre es
sencilla ya que para eliminar estos comportamientos uno primero debe ubicar la
fuente que lo produce. Aqu\'i es donde el tipado juega un rol importante ya que
en la etapa de compilaci\'on de nuetro programa el tipado nos ayuda a detectar
errores de codificaci\'on de manera temprana.

\section{Sem\'antica denotacional}
Antes mencionamos vagamente la interpretaci\'on de frases sintacticamente 
correctas, veamos de manera un poco mas precisa a que nos referimos. 
La manera de dar sem\'antica que vamos a utilizar en general va a ser 
denotacional, esta b\'asicamente nos define ecuaciones de manera de que
elementos sintacticos se correspondan con elementos sem\'anticos, algo que
surge es entonces es definir la correcta colecci\'on de elementos sem\'anticos,
con el fin de poder dar a cada elemento sintactico al menos un elementos sem\'antico.
