\chapter{Introducci\'on}
\label{chap:intro}

Como lo sugiere el titulo de la tesis, esta tratara sobre el estudio
de sem\'anticas categ\'oricas para lenguajes Algol-like. Para esto
vamos a estudiar tres lenguajes, $\lambdaarrow$, $\lambdaleq$ y $\Alike$,
este ultimo perteneciente a la clase de lenguajes Algol-like y para el cual
vamos a dar dos definiciones sem\'anticas.
La base de los lenguajes que vamos a estudiar se puede encontrar en los
cap\'itulos 15, 16 y 19 de \cite{reynolds2009theories} y \cite{olesfunctorcategories}, 
la idea de estudiar
categ\'oricamente esta en realidad motivado por la doble definici\'on
sem\'antica que pretendemos dar de $\Alike$. Vamos a querer mostrar
que podemos cambiar el significado
sem\'antico de un lenguaje sin cambiar sus ecuaciones, simplemente
eligiendo ciertas categor\'ias.\\

El trabajo empezar\'a definiendo el lenguaje $\lambdaarrow$, que es el
c\'alculo lambda simplemente tipado con constantes y algunos operadores.
Este lenguaje nos servir\'a para introducir conceptos generales sobre 
los lenguajes que vamos a estudiar, tales como
los contextos, la sem\'antica intr\'inseca para los tipos o la definici\'on
de sem\'antica para un juicio de tipado como manera de representar el significado
de las frases de nuestro lenguaje, entre otras cuestiones relacionas. 
Otro aporte de esta tesis es la implementaci\'on de cada uno de los lenguajes
a estudiar utilizando un lenguaje de programaci\'on con tipos dependientes 
\cite{idrislanguage}. 
En este sentido empezar con $\lambdaarrow$ nos sirvi\'o para
familiarizarnos con un lenguaje con tipos dependientes y esperamos tambi\'en
facilite la comprensi\'on del lector.\\

El siguiente paso va a ser tomar a $\lambdaarrow$ y extender su sistema de 
tipos para que incluya subtipado, a este nuevo lenguaje lo llamaremos
$\lambdaleq$. Algo interesante de este nuevo del lenguaje $\lambdaleq$
es que vamos a reutilizar completamente todas las definiciones que hicimos
en $\lambdaarrow$. A su vez, esta forma de ir enriqueciendo cada lenguaje 
dar\'a lugar a $\Alike$. Una ventaja de realizar las pruebas de una manera 
general, teniendo en cuenta esta jerarqu\'ia de lenguajes, es que las mismas
ser\'an validas, o servir\'an para completar las pruebas, para los
lenguajes mas expresivos.\\

Para finalizar vamos a extender el lenguaje $\lambdaleq$ con caracter\'isticas
imperativas, dando lugar al lenguaje $\Alike$. Este ser\'a un lenguaje 
Algol-like el cual combina aspectos imperativos y funcionales. Una particularidad
importante de estos lenguajes es que los mismo cuanta con una sem\'antica
que respeta la stack discipline. Gracias a dar una sem\'antica categ\'orica
para este lenguaje, en esta tesis mostramos como variando algunas elecciones
se puede tener una sem\'antica para $\Alike$ sin stack discipline.\\

El segundo cap\'itulo, puntualmente va a incluir la definici\'on del lenguaje
$\lambdaarrow$, esto ser\'a presentar la sint\'axis y las ecuaciones
sem\'anticas, adem\'as vamos a introducir todos los conceptos relacionados al 
sistemas de tipo simple, para finalizar vamos a dar pruebas de
continuidad de las ecuaciones sem\'anticas y vamos a probar la correcci\'on
del lenguaje con respecto a la regla-$\beta$. Para finalizar presentamos
una pequeña implementaci\'on en Idris.\\

En el tercer cap\'itulo vamos a presentar todos los conceptos sobre
un sistema de tipos con subtipado y adecuar entonces los tipos de $\lambdaarrow$
para definir el sistema de tipos de $\lambdaleq$, vamos a probar
continuidad y como propiedad interesante teniendo subtipado, vamos a probar
coherencia.\\

El cuarto cap\'itulo tratar\'a sobre el lenguaje $\Alike$, como en el segundo
cap\'itulo, vamos a presentar la sint\'axis y las ecuaciones sem\'anticas. 
Algo interesante ser\'a que ahora nuestra categor\'ia principal ser\'a
una categor\'ia funtorial. Para
este lenguaje, como ya mencionamos, vamos a dar dos sem\'anticas distintas, adem\'as
vamos a dar una prueba de la naturalidad de las ecuaciones sem\'anticas.\\

La tesis requerir\'a conocimientos previos sobre teor\'ia de dominios 
(predominios, dominios y funciones continuas), teor\'ia
de categor\'ias (categor\'ias concretas y funtoriales, producto, 
objeto exponencial y funtores) y sem\'antica denotacional. Un
conocimiento extra sobre tipos dependientes resultara practico exclusivamente
para la parte de implementaci\'on.
