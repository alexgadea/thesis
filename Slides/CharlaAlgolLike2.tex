\documentclass{beamer} % en min\'uscula!
\usefonttheme{professionalfonts} % fuentes de LaTeX
\usetheme{Warsaw} % tema escogido en este ejemplo
%%%% packages y comandos personales %%%%
\usepackage[latin1]{inputenc}
\usepackage{latexsym,amsmath,amssymb} % S\'imbolos
\usepackage{diagrams}
\usepackage{stmaryrd}
\usepackage{xcolor}
\usepackage{MnSymbol}
\usepackage{../Latex/thesis}

\begin{document}
\title{Estudiando sem\'anticas de Algol-Like.}
\author{{Ale Gadea.}\\
\vspace*{0.5cm}}
%\date{}
\frame{\titlepage}

\section{Introducci\'on}

\begin{frame}
\frametitle{Introducci\'on}
\begin{block}{Gu\'ia}
1. Definiciones iniciales.\\
2. Idea general.\\
3. Definiciones finales.\\
4. Sem\'antica del newvar.\\
5. Ejemplo, con SD vs sin SD.
\end{block}
\end{frame}

\begin{frame}
\frametitle{Definiciones iniciales}
\framesubtitle{Lenguaje}

\begin{block}{Gram\'atica}
$\lrangles{Phrase} ::=$ $\lrangles{PBool}$ $|$ $\lrangles{PInt}$ $|$ $\lrangles{PReal}$ \\
\ \ \ \ \ \ \ \ \ \ \ \ \ \ \ $|$ 
$\odot \lrangles{Phrase}$ $|$ $\lrangles{Phrase} \otimes \lrangles{Phrase} $ \\ 
\ \ \ \ \ \ \ \ \ \ \ \ \ \ \ $|$ 
$\assig{\lrangles{Phrase}}{\lrangles{Phrase}}$ $|$ $\cskip$ $|$ $\seq{\lrangles{Phrase}}{\lrangles{Phrase}}$ \\ 
\ \ \ \ \ \ \ \ \ \ \ \ \ \ \ $|$ 
$\cifthenelse{\lrangles{Phrase}}{\lrangles{Phrase}}{\lrangles{Phrase}}$ \\ 
\ \ \ \ \ \ \ \ \ \ \ \ \ \ \ $|$ 
$\cwhiledo{\lrangles{Phrase}}{\lrangles{Phrase}}$ \\ 
\ \ \ \ \ \ \ \ \ \ \ \ \ \ \ $|$ 
$\newdeltavar{\lrangles{Id}}{\lrangles{Phrase}}{\lrangles{Phrase}}$ \\ 
\ \ \ \ \ \ \ \ \ \ \ \ \ \ \ $|$ 
$\lrangles{Id}$ $|$ $\lrangles{Phrase}\lrangles{Phrase}$ \\
\ \ \ \ \ \ \ \ \ \ \ \ \ \ \ $|$ 
$\clambda{\lrangles{Id}}{\theta}{\lrangles{Phrase}} $ \\
\ \ \ \ \ \ \ \ \ \ \ \ \ \ \ $|$ 
$\rec{\lrangles{Phrase}}$\\
\

donde $\delta \in \lrangles{Data \ types}$, $\odot \in \{-, \neg\}$ y 
$\otimes \in \{+,-,*,/,\div,\rem,\wedge,\vee,\Rightarrow,\Leftrightarrow, =,\neq,<,>,\leq,\geq\}$\\
\end{block}

\end{frame}

\begin{frame}
\frametitle{Definiciones iniciales}
\framesubtitle{Tipos}

\begin{block}{Gram\'atica de tipos}
$\lrangles{data \ type} ::=$ $\intt$ $|$ $\realt$ $|$ $\boolt$\\
\

$\lrangles{phrase \ type}::=$ $\deltavar$ $|$ $\deltaexp$ $|$ $\deltaacc$ $|$ $\commt$\\
\ \ \ \ \ \ \ \ \ \ \ \ \ \ \ \ \ \ \ \ 
$|$ $\lrangles{phrase \ type} \rightarrow \lrangles{phrase \ type}$ \\
\

con $\delta \in \lrangles{data \ type}$\\
\end{block}

\begin{block}{Categor\'ia de tipos}
$\Theta_0$ $=$ $\{\theta \ | \ \theta \ \in \lrangles{Phrase \ types} \}$\\
$\Theta_1$ $=$ $\{\theta \rTo \theta' \ | \ \theta,\theta' \ \in \lrangles{Phrase \ types} \wedge \theta \leq \theta' \}$\\
\end{block}
\end{frame}

\begin{frame}
\frametitle{Definiciones iniciales}
\framesubtitle{Contextos}

\begin{block}{Gram\'atica de contextos}
$\lrangles{Context}::=$ $\varnothing$ $|$ $\lrangles{Context}, \lrangles{Pattern}:\lrangles{Phrase \ type}$
\end{block}

\begin{block}{Categor\'ia de contextos}
Sean $\pi$ y $\pi'$ dos contextos cualquiera, decimos que 
$\pi \leq \pi'$ cuando $dom(\pi)$ $=$ $dom(\pi')$ y $\forall \iota \in dom(\pi)$ 
$\pi \iota \leq \pi' \iota$. \\
\

$\Pi_0$ $=$ $\{\pi \ | \ \pi \ \in \lrangles{Context} \}$\\
$\Pi_1$ $=$ $\{\pi \rTo \pi' \ | \ \pi,\pi' \ \in \lrangles{Context} \wedge \pi \leq \pi' \}$\\
\end{block}
\end{frame}

\begin{frame}
\frametitle{Definiciones iniciales}
\framesubtitle{Categor\'ia de formas.}

\begin{block}{Categor\'ia de formas}
$\C_0$ $=$ colecci\'on de conjuntos de "determinada forma"\\
$\C_1$ $=$ $\{(h,s): C \rTo C' $ $|$  C y C' objetos de $\C_0 \}$ \\
\

donde $h : C' \rTo C$ y $s : (C \rTo C_\bot) \rTo (C' \rTo C'_\bot)$.\\
\end{block}

\end{frame}

\begin{frame}\small
\frametitle{Definiciones iniciales}
\framesubtitle{Sem\'antica de tipos}

\begin{block}{Funtor sobre tipos, $\semBrcks{ \ }$ $:$ $\Theta \rTo \DC$}
$\semBrcks{ \deltaexp }C$ $=$ $C \longrightarrow (S_{ \delta })_\bot$\\
$\semBrcks{ \deltaexp }(h,s)$ $:$ $\semBrcks{ \deltaexp }C \rTo \semBrcks{ \deltaexp }C'$\\
$\semBrcks{ \deltaexp }(h,s)e$ $=$ $e \circ h$ \\
\

$\semBrcks{ \deltaacc }C$ $=$ $S_{\delta} \longrightarrow \semBrcks{ \commt }C$\\
$\semBrcks{ \deltaacc }(h,s)$ $:$ $\semBrcks{ \deltaacc }C \rTo \semBrcks{ \deltaacc }C'$\\
$\semBrcks{ \deltaacc }(h,s)a$ $=$ $s \circ a$\\
\

$\semBrcks{ \commt }C$ $=$ $C \longrightarrow C_\bot$\\
$\semBrcks{ \commt }(h,s)$ $:$ $\semBrcks{ \commt }C \rTo \semBrcks{ \commt }C'$\\
$\semBrcks{ \commt }(h,s)$ $=$ $s$\\
\

$\semBrcks{ \deltavar }C$ $=$ $\semBrcks{ \deltaacc }C \times  \semBrcks{ \deltaexp }C$\\

\end{block}
con $\delta$ un $\lrangles{data \ type}$

\end{frame}

\begin{frame}\small
\frametitle{Definiciones iniciales}
\framesubtitle{Sem\'antica de tipos}

\begin{block}{Funtor sobre tipos, $\semBrcks{ \ }$ $:$ $\Theta \rTo \DC$}
$\semBrcks{ \theta \rightarrow \theta' }C$ $=$
$\prod\limits_{\widehat{C} \ en \ \C_0} 
(\semBrcks{ \theta }(C \concat \widehat{C}) \rightarrow  \semBrcks{ \theta' }(C \concat \widehat{C}))$ \\
\

$\semBrcks{ \theta \rightarrow \theta' }(h,s)$ 
$:$ $\semBrcks{ \theta \rightarrow \theta' }C \rTo \semBrcks{ \theta \rightarrow \theta' }C'$\\
$\semBrcks{ \theta \rightarrow \theta' }(h,s)f\widehat{C}$ $=$ $f(C' \concat \widehat{C})$\\
\

$\semBrcks{\intexp \leq \realexp}$ $:$ $\semBrcks{\intexp} \rTo \semBrcks{\realexp}$\\
$\semBrcks{\intexp \leq \realexp}C$ $=$ $\semBrcks{\intexp}C \rTo \semBrcks{\realexp}C$\\
$\semBrcks{\intexp \leq \realexp}Ce$ $=$ $\J \circ e$ \\
\

$\semBrcks{\realacc \leq \intacc}$ $:$ $\semBrcks{\realacc} \rTo \semBrcks{\intacc}$\\
$\semBrcks{\realacc \leq \intacc}C$ $=$ $\semBrcks{\realacc}C \rTo \semBrcks{\intacc}C$\\
$\semBrcks{\realacc \leq \intacc}Ca$ $=$ $a \circ \J$ \\
\

donde $\J$ es la inyecci\'on de enteros en reales.\\
\end{block}
\end{frame}

\begin{frame}
\frametitle{Definiciones iniciales}
\framesubtitle{Sem\'antica de contextos}

\begin{block}{Funtor sobre contextos}
$\semBrcks{ \ }$ $:$ $\Pi \rTo \DC$\\
\

$\semBrcks{\pi}C$ $=$ $\prod\limits_{\iota \in dom(\pi)} \semBrcks{\pi \iota}C$\\
\

$\semBrcks{\pi}(h,s)$ $:$ $\semBrcks{\pi}C \rTo \semBrcks{\pi'}C'$\\
$\semBrcks{\pi}(h,s)\eta\iota$ $=$ $\semBrcks{\pi \iota}(h,s)(\eta\iota)$ $\forall \iota \in dom(\pi)$\\

\end{block}
\end{frame}

\begin{frame}
\frametitle{Definiciones iniciales}
\framesubtitle{Sem\'antica de phrases}

\begin{block}{Transformaci\'on natural sobre un juicio de tipado}

$\semBrcks{\pi \vdash p : \theta}$ $:$ $\semBrcks{\pi} \rTo \semBrcks{\theta}$

\end{block}

\begin{block}{Sem\'antica de algunos comandos}\small
$\semBrcks{ \pi \vdash \cskip : \commt }C\eta\sigma$ 
$=$ $\iotabot \sigma$\\
\
			
$\semBrcks{ \pi \vdash \assig{e}{e'} : \commt }C\eta\sigma$ 
$=$ $(\lambda x . \semBrcks{\pi \vdash e : \deltaacc}C\eta x)_{\dbot}
(\semBrcks{ \pi \vdash e : \deltaexp }C\eta\sigma)$\\
\

$\semBrcks{ \pi \vdash \seq{e}{e'} : \commt }C\eta\sigma$ 
$=$ \\
\ \ \ \ $(\lambda \sigmahat . \semBrcks{\pi \vdash e : \commt}C\eta\sigmahat)_{\dbot}
(\semBrcks{ \pi \vdash e : \commt}C\eta\sigma)$\\
\end{block}

\end{frame}

\section{Pensando un rato}

\begin{frame}
\frametitle{Idea general}

\begin{block}{Comportamiento que buscamos}
$\semBrcks{\pi \vdash \newvar{\deltavar}{x}{p}{c} : \commt}C\eta\sigma$ $=$ $\sigma'$

contra

$\semBrcks{\pi \vdash \newvar{\deltavar}{x}{p}{c} : \commt}C\eta\sigma$ $=$ $\sigma$
\end{block}

\begin{block}{En particular}
$\semBrcks{ \vdash \newvar{\intvar}{x}{1}{\cskip} : \commt}\lrangles{\ }[]\lrangles{\ }$ $=$ $\lrangles{1}$

contra

$\semBrcks{ \vdash \newvar{\intvar}{x}{1}{\cskip} : \commt}\lrangles{\ }[]\lrangles{\ }$ $=$ $\lrangles{\ }$
\end{block}
\end{frame}

\begin{frame}\small
\frametitle{Definiciones finales}

\begin{block}{Categor\'ia de formas hasta ahora}
$\C_0$ $=$ colecci\'on de conjuntos de "determinada forma"\\
$\C_1$ $=$ $\{(h,s): C \rTo C' $ $|$  C y C' objetos de $\C_0 \}$ \\
\end{block}

\begin{block}{Stack discipline}
Un objeto $C$ sera un conjunto de secuencias pertenecientes a $S_1 \times \ldots \times S_n$, 
con $0 \leq n$ y $S_i$ cualquier conjunto. Diremos entonces que $C$ tiene forma $<S_1,\ldots,S_n>$.\\
Dados tres objetos $C$, $C'$, $C''$ y una flecha $(h,s)$ $:$ $C \rTo C'$ de $\C$ con $C' = C \concat C''$,\\
\

$h: C' \rTo C$\\
$h = head_{C}$\\
\

$s: (C \rTo C_\bot) \rTo (C' \rTo C'_\bot)$\\
$s \ c$ $=$ $\lambda \sigma' . (\lambda \sigma . \sigma \concat tail_{C''}\sigma')_{\dbot}(c(head_C \sigma'))$

\end{block}
\end{frame}

\begin{frame}
\frametitle{Definiciones finales}

\begin{block}{Categor\'ia de formas hasta ahora}
$\C_0$ $=$ colecci\'on de conjuntos de "determinada forma"\\
$\C_1$ $=$ $\{(h,s): C \rTo C' $ $|$  C y C' objetos de $\C_0 \}$ \\
\end{block}

\begin{block}{$\neg$ Stack discipline}
Definimos un \'unico objeto como la \'union de todos los conjuntos con forma anteriores.\\
Tomando este objeto $C$ de $\C$
\

$h: C \rTo C$\\
$h = 1_C$\\
\

$s: (C \rTo C_\bot) \rTo (C \rTo C_\bot)$\\
$s \ c$ $=$ $c$

\end{block}
\end{frame}

\begin{frame}\small
\frametitle{Sem\'antica del newvar con stack discipline}

\begin{block}{Comando newvar}
$\semBrcks{\pi \vdash \newvar{\deltavar}{x}{p}{c} : \commt}C\eta\sigma$ $=$
$(\lambda \sigma' . \ H \sigma')_{\dbot}$\\
$(\lambda \ v . \semBrcks{\pi,x:\deltavar \vdash c : \commt}(C\concat\lrangles{S_\delta})$\\
\ \ \ \ \ \ \ \ \ \ \ $[\semBrcks{\pi}(h,s)\eta \ | \ x:\lrangles{a_\sigma,e_\sigma}]$
							$(\sigma \concat \lrangles{v}))$
$(\semBrcks{\pi \vdash p : \deltaexp}C\eta\sigma)$\\
$(h,s)$ $:$ $C \rTo (C \concat \lrangles{S_\delta})$
\end{block}

\begin{block}{Con stack discipline}
$H = head_C$\\
$a_\sigma$ $=$ $\lambda x. \ \lambda \sigmahat . \ \iotabot ((head_{C} \sigmahat) \concat \lrangles{x})$\\
$e_\sigma$ $=$ $ \iotabot \circ last_{S_\delta}$
\end{block}

\begin{block}{Sin stack discipline}
$H = 1_C$\\
$a_\sigma \ v \ \sigmahat$ $=$ $ \lrangles{\sigmahat \ | \ (newRef \ \sigma):v} $\\
$e_\sigma \ \sigmahat$ $=$ $\sigmahat(newRef \ \sigma)$
\end{block}

\end{frame}

\section{Ejemplo}

\begin{frame}
\frametitle{Ejemplo}

\begin{block}{Programa ejemplo, p}
$\cnew$ $\intvar$ $\assig{x}{1}$ $\cin$\\
\ \ \ \ $\cskip$\\
$\cend$ $\textbf{;}$\\
$\cnew$ $\intvar$ $\assig{y}{2}$ $\cin$\\
\ \ \ \ $\assig{y}{0}$\\
$\cend$ 
\end{block}

\begin{block}{Ejecuci\'on con stack discipline}
$\semBrcks{\vdash p : \commt}\lrangles{}[]\lrangles{}$ $=$ $\lrangles{}$
\end{block}

\begin{block}{Ejecuci\'on sin stack discipline}
$\semBrcks{\vdash p : \commt}\lrangles{}[]\lrangles{}$ $=$ $\lrangles{r_0:1,r_1:0}$
\end{block}


\end{frame}

\end{document}
