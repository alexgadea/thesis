\documentclass{beamer}
\usepackage[spanish]{babel} % Para separar correctamente las palabras
\usepackage[utf8]{inputenc} % Este paquete permite poner acentos y eñes usando codificación utf-8

\usefonttheme{professionalfonts} % fuentes de LaTeX
\usetheme{Warsaw} % tema escogido en este ejemplo
\usecolortheme[named=red]{structure}
%%%% packages y comandos personales %%%%
\usepackage{latexsym,amsmath,amssymb} % S\'imbolos
\usepackage{diagrams}
\usepackage{stmaryrd}
\usepackage{xcolor}
\usepackage{MnSymbol}
\usepackage{../Latex/thesis}
\usepackage{bussproofs}

\begin{document}
\beamertemplatenavigationsymbolsempty
\title{Estudio de Semántica Categórica para Lenguajes Algol-Like}
\author{{Alejandro Emilio Gadea}\\
\vspace*{0.5cm}}
\date{}
\frame{\titlepage}

\section{Introducci\'on}

\begin{frame}
\frametitle{Introducci\'on}
\begin{block}{Gu\'ia}
1. Lenguaje $\lambdaarrow$\\
\quad \quad - Preliminares \\
\quad \quad - Continuidad \\
\quad \quad - Corrección regla-$\beta$\\

\pause

2. Lenguaje $\lambdaleq$\\
\quad \quad - Preliminares \\
\quad \quad - Coherencia \\

\pause

3. Lenguaje $\Alike$\\
\quad \quad - Preliminares \\
\quad \quad - Semántica con y sin stack discipline \\
\quad \quad - Naturalidad 

\end{block}
\end{frame}

\section{Lenguaje $\lambdaarrow$}

\begin{frame}
\frametitle{Lenguaje $\lambdaarrow$}
\framesubtitle{Preliminares}

\begin{block}{Lenguaje $\lambdaarrow$}

Este es un lenguaje funcional con tipado explicito simple, base
de lenguajes como Haskell o ML. Sus tipos son $\boolt$, $\intt$ y $\realt$ 
y tiene como frases relevantes a la aplicación, abstracción lambda, recursión, etc.

\end{block}

\end{frame}

\begin{frame}
\frametitle{Lenguaje $\lambdaarrow$}
\framesubtitle{Preliminares}

\begin{block}{Algunas reglas de inferencia}

\begin{center}
\AxiomC{}
\UnaryInfC{$\pi \vdash b : \boolt$}
\DisplayProof
\quad
\AxiomC{$\pi \vdash e : \boolt$}
\UnaryInfC{$\pi \vdash \neg e : \boolt$}
\DisplayProof
\end{center}

\pause

\begin{center}
\AxiomC{$\iota:\theta \in \pi$}
\RightLabel{}
\UnaryInfC{$\pi \vdash \iota : \theta$}
\DisplayProof
\quad
\AxiomC{$\pi,\iota:\theta \vdash e : \theta'$}
\UnaryInfC{$\pi \vdash \clambda{\iota}{\theta}{e} : \theta \rightarrow \theta'$}
\DisplayProof
\end{center}

\begin{center}
\AxiomC{$\pi \vdash e : \theta \rightarrow \theta$}
\UnaryInfC{$\pi \vdash \rec{e} : \theta$}
\DisplayProof
\quad
\AxiomC{$\pi \vdash e : \theta \rightarrow \theta'$}
\AxiomC{$\pi \vdash e': \theta$}
\BinaryInfC{$\pi \vdash ee' : \theta'$}
\DisplayProof
\end{center}

\begin{center}
\AxiomC{$\pi \vdash b  : \boolt$}
\AxiomC{$\pi \vdash e  : \theta$}
\AxiomC{$\pi \vdash e' : \theta$}
\TrinaryInfC{$\pi \vdash \cifthenelse{b}{e}{e'} : \theta$}
\DisplayProof
\end{center}

\end{block}

\end{frame}

\begin{frame}
\frametitle{Lenguaje $\lambdaarrow$}
\framesubtitle{Preliminares}

\begin{block}{Categorías para $\lambdaarrow$}

- $\Dom$, categoría de dominios y funciones continuas.\\
- $\Theta$, categoría discreta de $\lrangles{Type}$.\\
- $\Pi$, categoría discreta de contextos

\end{block}

\pause

\begin{block}{Funtores para $\lambdaarrow$}

$\semBrcks{ \ } : \Theta \rightarrow \Dom$ 
\quad \quad \quad \
$\semBrcks{ \ } : \Pi \rightarrow \Dom$\\
$\semBrcks{\delta}_0$ $=$ $(S_\delta)_\bot$
\quad \quad \quad \quad
$\semBrcks{\pi}_0$ $=$ $\prod\limits_{\iota \in dom \ \pi} \semBrcks{\pi\iota}$\\
$\semBrcks{\theta \rightarrow \theta'}_0$ $=$ $\semBrcks{\theta'}_0^{\semBrcks{\theta}_0}$

\end{block}

\pause

\begin{block}{Definición ecuación semántica}
$\semBrcks{\pi \vdash e : \theta}$ será un morfismo en la categoría $\Dom$ entre
$\semBrcks{\pi}$ y $\semBrcks{\theta}$
\end{block}

\end{frame}

\begin{frame}
\frametitle{Lenguaje $\lambdaarrow$}
\framesubtitle{Preliminares}

\begin{block}{Ecuaciones $\lambdaarrow$}
\begin{center}
$\semBrcks{\pi \vdash b : \boolt} \eta$ $=$ $\iotabot b$ \quad \quad \quad
$\semBrcks{\pi \vdash \neg e : \boolt}$ $=$ $\neg_{\odot} \circ \semBrcks{\pi \vdash e : \boolt}$
\end{center}

\begin{center}
$\semBrcks{\pi \vdash \iota : \theta}\eta$ $=$ $\eta \iota$
\end{center}

\begin{center}
$\semBrcks{\pi \vdash \clambda{\iota}{\theta}{e} : \theta \rightarrow \theta'}$ $=$
		$\semBrcks{\pi,\iota:\theta \vdash e : \theta'} \circ ext_{\iota,\theta}$
\end{center}

\begin{center}
$\semBrcks{\pi \vdash \rec{e} : \theta}\eta$ $=$ $\Y_{\theta} (\semBrcks{\pi \vdash e : \theta \rightarrow \theta}\eta)$
\end{center}

\begin{center}
$\semBrcks{\pi \vdash ee' : \theta'}\eta$ $=$ $\semBrcks{\pi \vdash e : \theta \rightarrow \theta'}\eta (\semBrcks{\pi \vdash e' : \theta}\eta)$
\end{center}

\end{block}

donde 
$ext_{\iota,\theta} : \semBrcks{\pi} 
					  \rightarrow \semBrcks{\theta} \rightarrow \semBrcks{\pi,\iota:\theta}$\\
\quad \quad \ \
$ext_{\iota,\theta} \eta \ x$ $=$ $[\ \eta \ | \ \iota:x \ ]$
\end{frame}

\begin{frame}
\frametitle{Lenguaje $\lambdaarrow$}
\framesubtitle{Continuidad}

\begin{block}{Teorema}
Dado un juicio de tipado $\pi \vdash e : \theta$ la ecuaci\'on sem\'antica
$\semBrcks{\pi \vdash e : \theta}$ es una funci\'on continua.
\end{block}

\pause

\begin{block}{}
Dada una cadena interesante
$\eta_0 \sqsubseteq \eta_1 \sqsubseteq \cdots$ en $\semBrcks{\pi}$, queremos ver

\begin{center}
$\semBrcks{\pi \vdash e : \theta}(\bigsqcup\limits^{\infty}_{i=0} \eta_i)$
$=$ $\bigsqcup\limits^{\infty}_{i=0} (\semBrcks{\pi \vdash e : \theta}\eta_i)$
\end{center}

\end{block}

\pause

\begin{block}{Nota}
Para probar el caso de la aplicación, fue útil utilizar la semántica 
categoría.
\end{block}

\end{frame}

\begin{frame}
\frametitle{Lenguaje $\lambdaarrow$}
\framesubtitle{Corrección regla-$\beta$}

\begin{block}{Teorema Corrección regla-$\beta$}
Dado un juicio de tipado $\pi \vdash (\clambda{\iota}{\theta}{e})e' : \theta'$ vale,

\begin{center}
$\semBrcks{\pi \vdash (\clambda{\iota}{\theta}{e})e' : \theta'}$ $=$
$\semBrcks{\pi \vdash e / \iota \leadsto e' : \theta'}$
\end{center}
\end{block}
\end{frame}

\begin{frame}

\begin{block}{Nueva forma de juicio de tipado y su regla de inferencia}
\begin{center}
\AxiomC{$\pi \vdash \Delta\iota : \pi'\iota$ \ \ \ para todo $\iota$ en $dom(\pi)$}
\UnaryInfC{$\pi \vdash \Delta : \pi'$}
\DisplayProof

\end{center}

donde $\Delta : \lrangles{Id} \rightarrow\lrangles{Phrase}$
\end{block}

\pause

\begin{block}{Teorema Substitución}

Sean $\pi$ y $\pi'$ contextos, $\Delta$ una substituci\'on y supongamos 

\begin{center}
$\pi' \vdash e : \theta$ \quad $\pi \vdash \Delta : \pi'$
\end{center}

donde  
$\eta'\iota$ $=$ $\semBrcks{\pi \vdash \Delta\iota : \pi'\iota}\eta$
$\forall \iota \in \ dom \ \pi'$, entonces\\

\begin{center}
$\semBrcks{\pi' \vdash e : \theta}\eta'$ $=$ $\semBrcks{\pi \vdash e / \Delta : \theta}\eta$.
\end{center}

\end{block}

\end{frame}

\section{Lenguaje $\lambdaleq$}

\begin{frame}
\frametitle{Lenguaje $\lambdaleq$}
\framesubtitle{Preliminares}

\begin{block}{Nueva forma de juicio de tipado}
Sean $\theta$, $\theta'$ tipos, diremos que $\theta$ es subtipo de $\theta'$ cuando

\begin{center}
$\theta \leq \theta'$
\end{center}

\end{block}

\pause

\begin{block}{Reglas de subtipado}

\begin{center}
\AxiomC{}
\UnaryInfC{$\theta \leq \theta$}
\DisplayProof
\quad
\AxiomC{$\theta \leq \theta'$}
\AxiomC{$\theta' \leq \theta''$}
\BinaryInfC{$\theta \leq \theta''$}
\DisplayProof
\quad
\AxiomC{$\theta_0 \leq \theta_0'$}
\AxiomC{$\theta_1 \leq \theta_1'$}
\BinaryInfC{$\theta_0' \rightarrow \theta_1 \leq \theta_0 \rightarrow \theta_1'$}
\DisplayProof
\end{center}

\end{block}

\begin{block}{Regla subsumption}

\begin{center}
\AxiomC{$\pi \vdash e : \theta$}
\AxiomC{$\theta \leq \theta'$}
\BinaryInfC{$\pi \vdash e : \theta'$}
\DisplayProof	
\end{center}

\end{block}

\end{frame}

\begin{frame}
\frametitle{Lenguaje $\lambdaleq$}
\framesubtitle{Preliminares}

\begin{block}{Subtipado para los tipos básicos}

\begin{center}
\AxiomC{}
\UnaryInfC{$\boolt \leq \intt$}
\DisplayProof
\quad
\AxiomC{}
\UnaryInfC{$\intt \leq \realt$}
\DisplayProof
\end{center}

\end{block}

\pause

\begin{block}{Categorías de $\lambdaleq$}

- $\Dom$, categoría de dominios y funciones continuas.\\
- $\Theta$, categoría de $\lrangles{Type}$ cuyos morfismos estarán determinados por $\leq$\\
- $\Pi$, categoría de contextos cuyos morfismos estarán determinados por $\leq$

\end{block}


\end{frame}

\begin{frame}
\frametitle{Lenguaje $\lambdaleq$}
\framesubtitle{Preliminares}

\begin{block}{Funtores para $\lambdaarrow$}

$\semBrcks{ \ } : \Theta \rightarrow \Dom$ 
\quad \quad \quad \quad \quad \quad \quad \quad \
$\semBrcks{ \ } : \Pi \rightarrow \Dom$ \\

$\semBrcks{\delta}_0$ $=$ $(S_\delta)_\bot$
\quad \quad \quad \quad \quad \quad \quad \quad \quad
$\semBrcks{\pi}_0$ $=$ $\prod\limits_{\iota \in dom \ \pi} \semBrcks{\pi\iota}$\\
$\semBrcks{\theta \rightarrow \theta'}_0$ $=$ $\semBrcks{\theta'}_0^{\semBrcks{\theta}_0}$
\quad \quad \quad \quad \quad \quad \pause
$\semBrcks{\pi \leq \pi'}_1$ $=$ $\prod\limits_{\iota \in dom \ \pi} \semBrcks{\pi\iota \leq \pi'\iota}$\\

\

$\semBrcks{\theta \leq \theta'}_{1} :
\semBrcks{\theta}_{0} \rightarrow \semBrcks{\theta'}_{0}$\\
$\semBrcks{\boolt \leq \intt}_1 \ x =$ if $x$ then $0$ else $1$\\
$\semBrcks{\intt \leq \realt}_1$ $=$ $\J_{\intt}^{\realt}$ \\
$\semBrcks{\theta \leq \theta}_1$ $=$ $1_{\semBrcks{\theta}}$\\
$\semBrcks{\theta \leq \theta''}_1$ $=$ $\semBrcks{\theta' \leq \theta''}_1 \circ \semBrcks{\theta \leq \theta'}_1$\\
$\semBrcks{(\theta_0 \rightarrow \theta'_0) \leq (\theta_1 \rightarrow \theta'_1)}_1 \ f$ 
$=$ $\semBrcks{\theta'_0 \leq \theta'_1}_1 \circ f \circ \semBrcks{\theta_1 \leq \theta_0}_1$

\end{block}

\end{frame}

\begin{frame}
\frametitle{Lenguaje $\lambdaleq$}
\framesubtitle{Preliminares}

\begin{block}{Ecuación para subsumption}
\begin{center}
$\semBrcks{\pi \vdash e : \theta'}$ $=$ $\semBrcks{\theta \leq \theta'} \circ \semBrcks{\pi \vdash e : \theta}$
\end{center}
\end{block}

\end{frame}

\begin{frame}
\frametitle{Lenguaje $\lambdaleq$}
\framesubtitle{Coherencia}

\begin{block}{Coherencia}
Sean $D$ y $D'$ derivaciones del juicio de tipado $\pi \vdash e : \theta$, entonces 
$\semBrcks{D} = \semBrcks{D'}$
\end{block}
\end{frame}

\begin{frame}
\frametitle{Lenguaje $\lambdaleq$}
\framesubtitle{Coherencia}

\begin{block}{Introduciendo coherencia (Ejemplo)}
Usando la regla del identificador:
\begin{center}
\AxiomC{$\iota : \realt \in \pi$}
\UnaryInfC{$\pi \vdash \iota : \realt$}
\DisplayProof
\end{center}

Usando Subsumption:
\begin{center}
\AxiomC{$\intt \leq \realt$}
\AxiomC{$\iota : \intt \in \pi$}
\UnaryInfC{$\pi \vdash \iota : \intt$}
\BinaryInfC{$\pi \vdash \iota : \realt$}
\DisplayProof
\end{center}

\end{block}

\end{frame}

\begin{frame}
\frametitle{Lenguaje $\lambdaleq$}
\framesubtitle{Coherencia}

\begin{block}{Introduciendo coherencia (Ejemplo)}
Usando la regla del identificador:
\begin{center}
\AxiomC{$\iota : \realt \in \pi$}
\UnaryInfC{$\pi \vdash \iota : \realt$}
\DisplayProof
\end{center}

Usando Subsumption:
\begin{center}
\AxiomC{$\intt \leq \realt$}
\AxiomC{$\iota : \intt \in \pi'$}
\UnaryInfC{$\pi' \vdash \iota : \intt$}
\BinaryInfC{$\pi' \vdash \iota : \realt$}
\DisplayProof
\end{center}

donde $\pi' \leq \pi$
\end{block}

\end{frame}

\begin{frame}
\frametitle{Lenguaje $\lambdaleq$}
\framesubtitle{Coherencia}

\begin{block}{Introduciendo coherencia (Ejemplo)}

Usando Subsumption$'$:
\begin{center}
\AxiomC{$\boolt \leq \intt$}
\AxiomC{$\intt \leq \realt$}
\BinaryInfC{$\boolt \leq \realt$}
\AxiomC{$\iota : \boolt \in \pi''$}
\UnaryInfC{$\pi'' \vdash \iota : \boolt$}
\BinaryInfC{$\pi'' \vdash \iota : \realt$}
\DisplayProof
\end{center}
\end{block}

\pause

\begin{block}{Lema de inversión}
Sea $\pi \vdash e : \theta$ un juicio de tipado valido cualquiera, entonces
la ultima regla aplicada es la regla referente a $e$ o subsumption.
\end{block}

\end{frame}

\begin{frame}
\frametitle{Lenguaje $\lambdaleq$}
\framesubtitle{Coherencia}

\begin{block}{Lema}

Sean $e$ una frase y $\theta$ un tipo cualesquiera y tal que tenemos 
juicios de tipado $\pi \vdash e : \theta$ y $\pi' \vdash e : \theta$
con $\eta : \semBrcks{\pi}$ y $\eta' : \semBrcks{\pi'}$ entonces vale

\begin{center}
$\semBrcks{\pi \vdash e : \theta}\eta = \semBrcks{\pi' \vdash e : \theta}\eta'$
\end{center}

\noindent
donde o bien pasa $\pi \leq \pi'$ y $\semBrcks{\pi \leq \pi'}\eta = \eta'$\\
\quad \quad \ \
o bien se da $\pi' \leq \pi$ y $\semBrcks{\pi' \leq \pi}\eta' = \eta$

\end{block}

\end{frame}

\begin{frame}
\frametitle{Lenguaje $\lambdaleq$}
\framesubtitle{Coherencia}

\begin{block}{Lema}

Sean $e$ una frase y $\theta$ un tipo cualesquiera y tal que tenemos 
juicios de tipado $\pi \vdash e : \theta$ y $\pi' \vdash e : \theta$
con $\eta : \semBrcks{\pi}$ y $\eta' : \semBrcks{\pi'}$ entonces vale

\begin{center}
$\semBrcks{\pi \vdash e : \theta}\eta = \semBrcks{\pi' \vdash e : \theta}\eta'$
\end{center}

\noindent
donde o bien pasa $\pi \leq \pi'$ y $\semBrcks{\pi \leq \pi'}\eta = \eta'$\\
\quad \quad \ \
o bien se da $\pi' \leq \pi$ y $\semBrcks{\pi' \leq \pi}\eta' = \eta$

\end{block}

\begin{block}{Coherencia}
Fijando ahora $\pi = \pi'$ obtenemos la propiedad de coherencia. Sean $D$ y $D'$
derivaciones del juicio de tipado $\pi \vdash e : \theta$, entonces 
$\semBrcks{D} = \semBrcks{D'}$
\end{block}

\end{frame}

\section{Lenguaje $\Alike$}

\begin{frame}
\frametitle{Lenguaje $\Alike$}
\framesubtitle{Preliminares}

\begin{block}{Essence of Algol}\small
\begin{enumerate}
\item Algol-like se obtiene de un lenguaje imperativo simple imponiendo un
sistema para los procedimientos basado en el calculo lambda ``fully typed'' y utilizando
call-by-name.

\item Existen dos clases de tipos: los $\textit{Data Types}$ y los $\textit{Phrase Types}$.

\item El orden de evaluaci\'on para las partes de una expresi\'on y su
conversi\'on impl\'icita deber\'ia estar indeterminada, pero el significado
del lenguaje es independiente de la indeterminaci\'on.

\item La definici\'on de procedimientos, recursi\'on, expresiones condicionales
pueden ser de cualquier $\textit{Phrase Types}$.

\item El lenguaje contiene stack discipline y su definici\'on debe hacer esta disciplina
obvia.
\end{enumerate}

\end{block}

\end{frame}

\begin{frame}
\frametitle{Lenguaje $\Alike$}
\framesubtitle{Preliminares}

\begin{block}{Sintaxis de los tipos}
$\lrangles{Data \ Types}$ ::= $\boolt$ $|$ $\intt$ $|$ $\realt$\\

\

$\lrangles{Phrase \ Types}$ ::= $\commt$\\
\quad \quad \quad \quad \quad 
\quad \quad \quad
$|$ $\deltaacc$\\
\quad \quad \quad \quad \quad 
\quad \quad \quad
$|$ $\deltaexp$ \\
\quad \quad \quad \quad \quad 
\quad \quad \quad
$|$ $\deltavar$\\
\quad \quad \quad \quad \quad 
\quad \quad \quad
$|$ $\lrangles{Phrase \ Types} \rightarrow \lrangles{Phrase \ Types}$
\end{block}

donde $\delta \in \lrangles{Data \ Types}$

\end{frame}

\begin{frame}
\frametitle{Lenguaje $\Alike$}
\framesubtitle{Preliminares}

\begin{block}{Sintaxis $\Alike$}
$\lrangles{Phrase}$ ::= $\lrangles{PBool}$ $|$ $\lrangles{PInt}$ $|$ $\lrangles{PReal}$\\
\quad \quad \quad \quad \quad
$|$ $\odot$ $\lrangles{Phrase}$ $|$ $\lrangles{Phrase}$ $\circledcirc$ $\lrangles{Phrase}$\\
\quad \quad \quad \quad \quad
$|$ $\textbf{if}$ $\lrangles{Phrase}$ $\textbf{then}$ $\lrangles{Phrase}$ $\textbf{else}$ $\lrangles{Phrase}$\\
\quad \quad \quad \quad \quad
$|$ $\lrangles{Id}$\\
\quad \quad \quad \quad  \quad
$|$ $\lrangles{Phrase}\lrangles{Phrase}$\\
\quad \quad \quad \quad  \quad
$|$ $\lambda$ $\lrangles{Id}_\theta$ . $\lrangles{Phrase}$\\
\quad \quad \quad \quad  \quad
$|$ $\textbf{rec} \lrangles{Phrase}$\\

\pause

\quad \quad \quad \quad  \quad
$|$ $\cnewv$ $\deltavar$ $\lrangles{Id}$ $\cassig$ $\lrangles{Phrase}$ $\cin$ $\lrangles{Phrase}$\\
\quad \quad \quad \quad  \quad
$|$ $\lrangles{Phrase}$ $\cassig$ $\lrangles{Phrase}$ $|$ $\cskip$ $|$ $\lrangles{Phrase}$ $\concatdots$ $\lrangles{Phrase}$\\
\quad \quad \quad \quad  \quad
$|$ $\cwhile$ $\lrangles{Phrase}$ $\cdo$ $\lrangles{Phrase}$
\end{block}

\end{frame}

\begin{frame}
\frametitle{Lenguaje $\Alike$}
\framesubtitle{Preliminares}

\begin{block}{Reglas de inferencia para los Phrase Types}

\begin{center}
\AxiomC{}
\UnaryInfC{$\intexp \leq \realexp$}
\DisplayProof
\quad \quad \quad
\AxiomC{}
\UnaryInfC{$\realacc \leq \intacc$}
\DisplayProof

\quad

\AxiomC{}
\UnaryInfC{$\deltavar \leq \deltaexp$}
\DisplayProof
\quad \quad \quad
\AxiomC{}
\UnaryInfC{$\deltavar \leq \deltaacc$}
\DisplayProof
\end{center}

\end{block}

\pause

\begin{block}{Algunas reglas de inferencia}

\begin{center}
\AxiomC{}
\UnaryInfC{$\pi \vdash b : \boolexp$}
\DisplayProof
\quad
\AxiomC{$\pi \vdash e : \boolexp$}
\UnaryInfC{$\pi \vdash \neg e : \boolexp$}
\DisplayProof
\end{center}

\begin{center}
\AxiomC{$\pi \vdash b  : \boolexp$}
\AxiomC{$\pi \vdash e  : \theta$}
\AxiomC{$\pi \vdash e' : \theta$}
\TrinaryInfC{$\pi \vdash \cifthenelse{b}{e}{e'} : \theta$}
\DisplayProof

\

con $\theta \in \{\deltavar,\deltaacc,\deltaexp,\commt\}$.\\
\end{center}
\end{block}

\end{frame}

\begin{frame}
\frametitle{Lenguaje $\Alike$}
\framesubtitle{Preliminares}

\begin{block}{Reglas de inferencia de la parte imperativa}

\begin{center}
\AxiomC{$\pi \vdash b : \boolexp$}
\AxiomC{$\pi \vdash e : \commt$}
\BinaryInfC{$\pi \vdash \cwhiledo{b}{e} : \commt$}
\DisplayProof
\quad
\AxiomC{}
\UnaryInfC{$\pi \vdash \cskip : \commt$}
\DisplayProof

\quad

\quad

\AxiomC{$\pi \vdash e  : \commt$}
\AxiomC{$\pi \vdash e' : \commt$}
\BinaryInfC{$\pi \vdash \cseq{e}{e'} : \commt$}
\DisplayProof

\quad

\quad

\AxiomC{$\pi \vdash e  : \deltaexp$}
\AxiomC{$\pi,\iota:\deltavar \vdash e' : \commt$}
\BinaryInfC{$\pi \vdash \newdeltavar{\iota}{e}{e'} : \commt$}
\DisplayProof

\quad

\quad

\AxiomC{$\pi \vdash e   : \deltaacc$}
\AxiomC{$\pi \vdash e'  : \deltaexp$}
\BinaryInfC{$\pi \vdash \assig{e}{e'} : \commt$}
\DisplayProof
\end{center}

\end{block}
\end{frame}

\begin{frame}
\frametitle{Lenguaje $\Alike$}
\framesubtitle{Preliminares}

\begin{block}{Categorías de $\Alike$}

- $\Theta$, categoría de $\lrangles{Phrase \ Types}$ 
cuyos morfismos estarán determinados por $\leq$\\
- $\Pi$, categoría de contextos cuyos morfismos estarán determinados por $\leq$

\end{block}


\end{frame}

\begin{frame}
\frametitle{Lenguaje $\Alike$}
\framesubtitle{Semántica con y sin stack discipline}

\begin{block}{Estado para la semántica con stack discipline}

Un estado será un elemento perteneciente a $S_1 \times \ldots \times S_n$, 
con $1 \leq n$ con $S_i$ cualquiera de los conjuntos $S_{\intt}$, $S_{\realt}$ o $S_{\boolt}$.\\
Diremos además que este estado tiene forma $\lrangles{S_1,\ldots,S_n}$

\end{block}

\pause

\begin{block}{Categorías para la semántica con stack discipline}

$\SD_0$ $=$ $\{C$ $|$ $C$ es el conjunto de todos los estados con determinada forma$\}$\\
$\SD_1(C,C')$ $=$ $\{C \rTo^{(h,s)} C'$ $|$ $C'$ extiende a $C\}$, $C' = C \concat \overline{C}$\\

\

donde \ $h: C' \rightarrow C$
\quad \quad \quad
	  $s: (C \rightarrow C_{\bot}) \rightarrow (C' \rightarrow C'_{\bot})$\\
\quad \quad \ \ \ 
	  $h = head_C$
\quad \quad \quad 
	  $s \ c \ \sigmahat =$ 
	  	   $(\lambda \sigma. \ \sigma \concat (tail_{\overline{C}} \ \sigmahat))_{\bot}$
	  	   	$(c(h \ \sigmahat))$\\
\end{block}

\end{frame}

\begin{frame}
\frametitle{Lenguaje $\Alike$}
\framesubtitle{Semántica con y sin stack discipline}

\begin{block}{Estado para la semántica sin stack discipline}

Un estado ser\'a un elemento perteneciente a 
$\Sigma = \bigcup\limits_{I \in \FinId} I \rightarrow S$, 
donde $\FinId$ ser\'a un conjunto de subconjuntos finitos de $\lrangles{Id}$ y
$S = S_{\intt} + S_{\realt} + S_{\boolt}$\\

\end{block}

\pause

\begin{block}{Categorías para la semántica sin stack discipline}
$\SSD_0$ $=$ $\{\ \Sigma \ \}$\\

\

$(1_h,1_s) : \Sigma \rightarrow \Sigma$, tal que\\

$1_h : \Sigma \rightarrow \Sigma$ \ \ \ \ \ \ \
$1_s: (\Sigma \rightarrow \Sigma_{\bot}) \rightarrow (\Sigma \rightarrow \Sigma_{\bot})$\\
\indent
$1_h \sigma = \sigma$ \ \ \ \ \ \ \ \ \ \ \ $1_s \ c = c$

\end{block}

\end{frame}

\begin{frame}
\frametitle{Lenguaje $\Alike$}
\framesubtitle{Semántica con y sin stack discipline}

\begin{block}{Categoría semántica}
$\PDom^{\C}$, categoría de funtores $\C \rightarrow \PDom$ y transformaciones naturales.
Donde $\PDom$ es la categoría de predominios y funciones continuas. Además con $\C$ vamos a
referirnos a $\SD$ o $\SSD$.
\end{block}

\pause

\begin{block}{Funtores para $\Alike$}
$\semBrcks{ \ } : \Theta \rightarrow \PDom^{\C}$\\
\pause
Dado un $\theta$ objeto de $\Theta$, $\semBrcks{\theta}_{0}$ será un funtor 
de $\C$ en $\PDom$\\
\

\pause
$\semBrcks{\theta}_{0}C : \PDom$, con $C$ objeto de $\C$\\
\pause
$\semBrcks{\theta}_{0}(h,s) : \semBrcks{\theta}_{0}C \rightarrow \semBrcks{\theta}_{0}C'$, con 
$C \rTo^{(h,s)} C'$ flecha de $\C$\\

\

$\semBrcks{\theta \rTo^{\leq} \theta'}_{1} : 
	\semBrcks{\theta}_{0} \rightarrow \semBrcks{\theta'}_{0}$\\

\end{block}

\end{frame}

\begin{frame}
\frametitle{Lenguaje $\Alike$}
\framesubtitle{Semántica con y sin stack discipline}

\begin{block}{Funtores para $\Alike$}

$\semBrcks{\deltaexp}_{0}C = C \rightarrow (S_{\delta})_{\bot}$
\quad \quad \quad \quad \quad 
$\semBrcks{\deltaacc}_{0}C = S_{\delta} \rightarrow \semBrcks{\commt}C$\\
$\semBrcks{\deltaexp}_{0} \ (h,s) \ e = e \circ h$
\quad \quad \quad \quad \quad 
$\semBrcks{\deltaacc}_{0} \ (h,s) \ a = s \circ a$\\

\

$\semBrcks{\commt}_{0}C = C \rightarrow C_{\bot}$\\
$\semBrcks{\commt}_{0} \ (h,s) \ c = s \ c$ \\

\

$\semBrcks{\deltavar}_{0}C = \semBrcks{\deltaacc}C \times \semBrcks{\deltaexp}C$\\
$\semBrcks{\deltavar}_{0} \ (h,s) \ (a,e) = 
			(\semBrcks{\deltaacc}(h,s)a,\semBrcks{\deltaacc}(h,s)e)$\\
			
\

$\semBrcks{\theta \rightarrow \theta'}_0 C$ $=$ 
					$\Nat{\semBrcks{\theta}(C \concat \_)}{\semBrcks{\theta'}(C \concat \_)}$\\
\indent
$\semBrcks{\theta \rightarrow \theta'}_0 \ (h,s) \ f \ \widehat{C}$ $=$ 
														$f(\overline{C} \concat \widehat{C})$\\
\indent \indent donde $C \concat \overline{C} = C'$\\

\end{block}

\end{frame}

\begin{frame}
\frametitle{Lenguaje $\Alike$}
\framesubtitle{Semántica con y sin stack discipline}

\begin{block}{Funtores para $\Alike$}

$\semBrcks{\intexp \leq \realexp}_1 C \ e$ $=$ $\J \circ e$
\quad 
$\semBrcks{\realacc \leq \intacc}_1 C \ a$ $=$ $a \circ \J$\\

\

$\semBrcks{\deltavar \leq \deltaexp}_1 C \ (a,e)$ $=$ $e$
\quad \quad \quad 
$\semBrcks{\deltavar \leq \deltaacc}_1 C \ (a,e)$ $=$ $a$\\

\


$\semBrcks{\theta \leq \theta}_1 C$ $=$ $1_{\semBrcks{\theta} C}$
\quad \quad \quad \quad 
$\semBrcks{\theta \leq \theta''}_1 C$ $=$ 
					$\semBrcks{\theta' \leq \theta''}_1 C \circ \semBrcks{\theta \leq \theta'}_1 C$\\

\

$\semBrcks{(\theta_0 \rightarrow \theta'_0) \leq (\theta_1 \rightarrow \theta'_1)}_1 C \ f \ \widehat{C}$ \\
\quad \quad \quad \quad \quad 
			$=$ 
			$\semBrcks{\theta'_0 \leq \theta'_1}_1 (C \concat \widehat{C}) 
				\circ 
			(f \ \widehat{C}) 
				\circ 
			\semBrcks{\theta_1 \leq \theta_0}_1 (C \concat \widehat{C}) 
			$\\

\end{block}

\end{frame}

\begin{frame}
\frametitle{Lenguaje $\Alike$}
\framesubtitle{Semántica con y sin stack discipline}

\begin{block}{Funtores para $\Alike$}

$\semBrcks{ \ } : \Pi \rightarrow \PDom^{\C}$\\
Dado un $\pi$ objeto de $\Pi$, $\semBrcks{\pi}_{0}$ será un funtor 
de $\C$ en $\PDom$\\
\

$\semBrcks{\pi}_{0}C : \PDom$, con $C$ objeto de $\C$\\
$\semBrcks{\pi}_{0}(h,s) : \semBrcks{\pi}_{0}C \rightarrow \semBrcks{\pi}_{0}C'$, con 
$C \rTo^{(h,s)} C'$ flecha de $\C$\\

\

$\semBrcks{\pi \rTo^{\leq} \pi'}_{1} : 
	\semBrcks{\pi}_{0} \rightarrow \semBrcks{\pi'}_{0}$\\

\end{block}

\end{frame}

\begin{frame}
\frametitle{Lenguaje $\Alike$}
\framesubtitle{Semántica con y sin stack discipline}

\begin{block}{Funtores para $\Alike$}

$\semBrcks{\pi}_0C$ $=$ $\prod\limits_{\iota \in dom \ \pi} \semBrcks{\pi\iota}C$\\

\

\

$\semBrcks{\pi}_0(h,s) \ \eta$ $=$ $\prod\limits_{\iota \in dom \ \pi} 
												\semBrcks{\pi\iota}(h,s)(\eta\iota)$\\
\

\

$\semBrcks{\pi \leq \pi'}_1C$ $=$ $\prod\limits_{\iota \in dom \ \pi} 
												\semBrcks{\pi\iota \leq \pi'\iota}C(\eta\iota)$

\end{block}

\pause

\begin{block}{Ecuación semántica $\Alike$}
$\semBrcks{\pi \vdash e : \theta}$ sera una transformación natural de $\semBrcks{\pi}$
en $\semBrcks{\theta}$
\end{block}

\end{frame}

\begin{frame}
\frametitle{Lenguaje $\Alike$}
\framesubtitle{Semántica con y sin stack discipline}

\begin{block}{Ecuaciones $\Alike$}
$\semBrcks{ \pi \vdash b : \boolexp }C \ \eta \ \sigma$ $=$ $\iotabot \ b$\\

\

$\semBrcks{ \pi \vdash \neg b : \boolexp }C$ 
$=$ 
$\neg_{\odot} \circ \semBrcks{\pi \vdash b : \boolexp}C$\\

\

$\semBrcks{ \pi \vdash \cifthenelse{b}{e}{e'}: \deltaacc}C \ \eta \ z \ \sigma$ 
$=$ \\
\quad \quad \quad 
$(\lambda b . \ if \ b $ $then \ \semBrcks{ \pi \vdash e : \deltaacc}C \ \eta \ z \ \sigma$\\
\quad \quad \quad \quad \quad \quad \quad 
$else \ \semBrcks{ \pi \vdash e' : \deltaacc}C \ \eta \ z \ \sigma)_{\dbot}$ \\
\quad \quad \quad \quad \quad \quad \quad \quad \quad \quad \quad \quad \quad \quad \quad 
$(\semBrcks{ \pi \vdash b : \boolexp}C \ \eta \ \sigma)$\\

\end{block}

\end{frame}

\begin{frame}

\frametitle{Lenguaje $\Alike$}
\framesubtitle{Semántica con y sin stack discipline}

\begin{block}{Ecuaciones $\Alike$}

$\semBrcks{ \pi \vdash \assig{e}{e'} : \commt }C \ \eta \ \sigma$ 
$=$ \\
\quad \quad \quad \quad \quad 
$(\lambda x . \ \semBrcks{\pi \vdash e : \deltaacc}C \ \eta \ x \ \sigma)_{\dbot}
(\semBrcks{ \pi \vdash e' : \deltaexp }C \ \eta \ \sigma)$\\

\

$\semBrcks{ \pi \vdash ee' : \theta' }C \ \eta$ 
$=$ 
$\semBrcks{ \pi \vdash e : \theta \rightarrow \theta' }C \ \eta \ \lrangles{} \
(\semBrcks{ \pi \vdash e' : \theta }C \ \eta)$\\

\

$\semBrcks{ \pi \vdash \lambda \iota_{\theta} . \ e : \theta \rightarrow \theta' }C \ \eta \ 
\overline{C} \ z$ 
$=$ \\
\quad \quad \quad \quad \quad \quad 
$\semBrcks{ \pi,\iota:\theta \vdash e : \theta' } \ (C \concat \overline{C}) \
[ \ \semBrcks{\pi}(h,s) \ \eta \ | \ \iota:z \ ]$\\

\

con $(h,s)$ $:$ $C \rTo (C \concat \overline{C})$ 

\end{block}

\end{frame}

\begin{frame}

\frametitle{Lenguaje $\Alike$}
\framesubtitle{Semántica con y sin stack discipline}

\begin{block}{Ecuación $\cnew \deltavar$ con stack discipline}

$\semBrcks{\pi \vdash \newdeltavar{\iota}{ei}{c} : \commt}C \ \eta \ \sigma$ 
$=$ $H_{\bot} ($\\ 
\quad \quad 
$(\lambda \ v . \ \semBrcks{\pi,\iota:\deltavar \vdash c : \commt}
		\ (C\concat\lrangles{S_\delta}) \ \eta_{ext} \ \sigma_{ext})_{\dbot}$ \\ 
\quad \quad \quad \quad \quad \quad \quad \quad \quad \quad 
\quad \quad \quad \quad \quad \quad \quad 
$(\semBrcks{\pi \vdash ei : \deltaexp}C \ \eta \ \sigma))$\\

\

con $(h,s)$ $:$ $C \rTo (C \concat \lrangles{S_\delta})$ \\
\quad \quad 
$a$ $=$ $\lambda z. \ \lambda \sigmahat . \ \iotabot ((head_{C} \sigmahat) \concat \lrangles{z})$\\
\quad \quad 
$e$ $=$ $ \iotabot \circ last_{S_\delta}$\\
\quad \quad 
$\sigma_{ext} = \sigma \concat \lrangles{v} $\\
\quad \quad 
$\eta_{ext} = [ \ \semBrcks{\pi}(h,s) \ \eta \ | \ \iota:\lrangles{a,e} \ ]$\\
\quad \quad 
$H = head_C$

\end{block}

\end{frame}

\begin{frame}

\frametitle{Lenguaje $\Alike$}
\framesubtitle{Semántica con y sin stack discipline}

\begin{block}{Ecuación $\cnew \deltavar$ sin stack discipline}

$\semBrcks{\pi \vdash \newdeltavar{\iota}{ei}{c} : \commt}C \ \eta \ \sigma$ 
$=$ $H_{\bot} ($ \\ 
\quad \quad
$(\lambda \ v . \ \semBrcks{\pi,\iota:\deltavar \vdash c : \commt}
C \ \eta_{new} \ \sigma_{new})_{\dbot}$\\
\quad \quad \quad \quad \quad \quad \quad \quad \quad \quad 
\quad \quad \quad \quad \quad \quad \quad 
$(\semBrcks{\pi \vdash ei : \deltaexp}C \ \eta \ \sigma))$\\

\

con 
$a \ v \ \sigma$ $=$ $\iotabot [ \ \sigma \ | \ \iota:\iotadelta v \ ] $\\
\quad \quad 
$e \ \sigma$ $=$ $(\lambda v . v)_{\delta}(\sigma \iota)$\\
\quad \quad 
$\sigma_{new} = [ \ \sigma \ | \ \iota:\iotadelta v \ ]$\\
\quad \quad 
$\eta_{new} = [ \ \eta \ | \ \iota:\lrangles{a,e} \ ]$\\
\quad \quad 
$H = 1_C$\\

\end{block}

\end{frame}

\begin{frame}
\frametitle{Lenguaje $\Alike$}
\framesubtitle{Naturalidad}

\begin{block}{Teorema de Naturalidad}
Dado un juicio de tipado valido $\pi \vdash e : \theta$ y una flecha 
$C \rTo{(h,s)} C'$ en $\C$ el siguiente diagrama conmuta,
\begin{center}
\begin{diagram}
   \semBrcks{\pi}C & & \rTo^{\semBrcks{\pi \vdash e : \theta}C} & & \semBrcks{\theta}C \\
   \dTo^{\semBrcks{\pi}(h,s)} & & & & \dTo_{\semBrcks{\theta}(h,s)} & \\
   \semBrcks{\pi}C' & & \rTo_{\semBrcks{\pi \vdash e : \theta}C'} & & \semBrcks{\theta}C' &
\end{diagram}
\end{center}

\
\end{block}

\end{frame}

\begin{frame}
\frametitle{Trabajos futuros}

\begin{block}{}

- Completar la ecuación semántica de \textbf{if}\\
\

- Tomar $\Alike$ y agregar intersección\\
\

- Probar coherencia para $\Alike$\\
\

- Terminar Forsythe\\

\end{block}

\end{frame}

\begin{frame}
\begin{center}\Huge
Gracias!!!
\end{center}
\end{frame}


\begin{frame}
\frametitle{Lenguaje $\lambdaarrow$}
\framesubtitle{Continuidad}

\begin{block}{Ecuación semántica aplicación}
Si tenemos

\begin{center}
$\semBrcks{\pi \vdash ee' : \theta'} : \semBrcks{\pi} \rightarrow \semBrcks{\theta'}$\\
$\semBrcks{\pi \vdash ee' : \theta'}\eta$ $=$ $\semBrcks{\pi \vdash e : \theta \rightarrow \theta'}\eta (\semBrcks{\pi \vdash e' : \theta}\eta)$
\end{center}

y definimos

\begin{center}
$f : \semBrcks{\pi} \times \semBrcks{\theta} \rightarrow \semBrcks{\theta'}$
\quad \quad \quad \quad \quad \
$g : \semBrcks{\pi} \rightarrow \semBrcks{\theta}$\\
$f = \uncurry (\semBrcks{\pi \vdash e : \theta \rightarrow \theta'})$
\quad
$g = \semBrcks{\pi \vdash e' : \theta}$
\end{center}

\end{block}

\end{frame}

\begin{frame}
\frametitle{Lenguaje $\lambdaarrow$}
\framesubtitle{Continuidad}

\begin{block}{Ecuación semántica aplicación}

\begin{diagram}
  \semBrcks{\theta'}^{\semBrcks{\theta}} \times \semBrcks{\theta} & \rTo^{\epsilon} & \semBrcks{\theta'}\\
  \uTo^{\widetilde{f} \times 1_{\semBrcks{\theta}}} & \ruTo^{f} \ruTo(2,4) & &  \\
  \semBrcks{\pi} \times \semBrcks{\theta} & & f \circ \lrangles{1_{\semBrcks{\pi}} , g} & \\
  \uTo^{\lrangles{1_{\semBrcks{\pi}} , g}}  \ruTo(3,2) & & &  \\
  \semBrcks{\pi} & & &
\end{diagram}

\end{block}

\end{frame}

\begin{frame}
\frametitle{Lenguaje $\lambdaarrow$}
\framesubtitle{Continuidad}

\begin{block}{Ecuación semántica aplicación}
Luego podemos definir

\begin{center}
$\semBrcks{\pi \vdash ee' : \theta'} : \semBrcks{\pi} \rightarrow \semBrcks{\theta'}$\\

\

$\semBrcks{\pi \vdash ee' : \theta'} = 
\epsilon \circ \lrangles{\semBrcks{\pi \vdash e : \theta \rightarrow \theta'}
						, \semBrcks{\pi \vdash e' : \theta}}$
\end{center}

además, tomando un $\eta$ en $\semBrcks{\pi}$\\

\

$\semBrcks{\pi \vdash ee' : \theta'}\eta = 
(\epsilon \circ \lrangles{\semBrcks{\pi \vdash e : \theta \rightarrow \theta'}
						, \semBrcks{\pi \vdash e' : \theta}})\eta$\\
\quad \quad \quad \quad \quad \quad 
$= \semBrcks{\pi \vdash e : \theta \rightarrow \theta'}\eta(\semBrcks{\pi \vdash e' : \theta}\eta)$

\end{block}

\end{frame}

\end{document}
