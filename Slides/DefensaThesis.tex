\documentclass{beamer}
\usepackage[spanish]{babel} % Para separar correctamente las palabras
\usepackage[utf8]{inputenc} % Este paquete permite poner acentos y eñes usando codificación utf-8

\usefonttheme{professionalfonts} % fuentes de LaTeX
\usetheme{Warsaw} % tema escogido en este ejemplo
\usecolortheme[named=red]{structure}
%%%% packages y comandos personales %%%%
\usepackage{latexsym,amsmath,amssymb} % S\'imbolos
\usepackage{diagrams}
\usepackage{stmaryrd}
\usepackage{xcolor}
\usepackage{MnSymbol}
\usepackage{../Latex/thesis}
\usepackage{bussproofs}

\begin{document}
\beamertemplatenavigationsymbolsempty
\title{Estudio de Semántica Categórica para Lenguajes Algol-Like}
\author{{Alejandro Emilio Gadea}\\
\vspace*{0.5cm}}
\date{}
\frame{\titlepage}

\section{Introducci\'on}

\begin{frame}
\frametitle{Introducci\'on}
\begin{block}{Gu\'ia}
1. Lenguaje $\lambdaarrow$\\
\quad \quad - Preliminares \\
\quad \quad - Continuidad \\
\quad \quad - Corrección regla-$\beta$\\

\pause

2. Lenguaje $\lambdaleq$\\
\quad \quad - Preliminares \\
\quad \quad - Coherencia \\

\pause

3. Lenguaje $\Alike$\\
\quad \quad - Preliminares \\
\quad \quad - Semántica con y sin stack discipline \\
\quad \quad - Naturalidad 

\end{block}
\end{frame}

\section{Lenguaje $\lambdaarrow$}

\begin{frame}
\frametitle{Lenguaje $\lambdaarrow$}
\framesubtitle{Preliminares}

\begin{block}{Sintaxis de los tipos}
$\lrangles{Type}$ ::= $\boolt$ $|$ $\intt$ $|$ $\realt$\\
\quad \quad \quad \quad
$|$ $\lrangles{Type} \rightarrow \lrangles{Type}$\\
\end{block}

\begin{block}{Sintaxis $\lambdaarrow$}
$\lrangles{Phrase}$ ::= $\lrangles{PBool}$ $|$ $\lrangles{PInt}$ $|$ $\lrangles{PReal}$\\
\quad \quad \quad \quad \quad
$|$ $\odot$ $\lrangles{Phrase}$ $|$ $\lrangles{Phrase}$ $\circledcirc$ $\lrangles{Phrase}$\\
\quad \quad \quad \quad \quad
$|$ $\textbf{if}$ $\lrangles{Phrase}$ $\textbf{then}$ $\lrangles{Phrase}$ $\textbf{else}$ $\lrangles{Phrase}$\\
\quad \quad \quad \quad \quad
$|$ $\lrangles{Id}$\\
\quad \quad \quad \quad  \quad
$|$ $\lrangles{Phrase}\lrangles{Phrase}$\\
\quad \quad \quad \quad  \quad
$|$ $\lambda$ $\lrangles{Id}_\theta$ . $\lrangles{Phrase}$\\
\quad \quad \quad \quad  \quad
$|$ $\textbf{rec} \lrangles{Phrase}$\\
\end{block}

\end{frame}

\begin{frame}
\frametitle{Lenguaje $\lambdaarrow$}
\framesubtitle{Preliminares}

\begin{block}{Contextos}

$\lrangles{Context}$ ::= $\varnothing$ $|$ $\lrangles{Context}$,$\lrangles{Id}$:$\lrangles{Type}$\\

\

\noindent
tal que dado cualquier contexto $\iota_0:\theta_0,\ldots,\iota_n:\theta_n$, los
identificadores $\iota_0,\ldots,\iota_n$ son todos distintos.
\end{block}

\begin{block}{Juicio de tipado}

Un juicio de tipado sera una relaci\'on entre un contexto $\pi$, una
frase del lenguaje $e$ y un tipo $\theta$ que establecer\'a que
bajo el contexto $\pi$ la expresi\'on $e$ tiene tipo $\theta$.
Lo denotaremos como  $\pi \vdash e : \theta$, 
en particular, cuando $\pi = \varnothing$ escribiremos
$\vdash e : \theta$

\end{block}

\end{frame}


\begin{frame}
\frametitle{Lenguaje $\lambdaarrow$}
\framesubtitle{Preliminares}

\begin{block}{Algunas reglas de inferencia}

\begin{center}
\AxiomC{}
\UnaryInfC{$\pi \vdash b : \boolt$}
\DisplayProof
\quad
\AxiomC{$\pi \vdash e : \boolt$}
\UnaryInfC{$\pi \vdash \neg e : \boolt$}
\DisplayProof
\end{center}

\pause

\begin{center}
\AxiomC{$\iota:\theta \in \pi$}
\RightLabel{}
\UnaryInfC{$\pi \vdash \iota : \theta$}
\DisplayProof
\quad
\AxiomC{$\pi,\iota:\theta \vdash e : \theta'$}
\UnaryInfC{$\pi \vdash \clambda{\iota}{\theta}{e} : \theta \rightarrow \theta'$}
\DisplayProof
\end{center}

\begin{center}
\AxiomC{$\pi \vdash e : \theta \rightarrow \theta$}
\UnaryInfC{$\pi \vdash \rec{e} : \theta$}
\DisplayProof
\quad
\AxiomC{$\pi \vdash e : \theta \rightarrow \theta'$}
\AxiomC{$\pi \vdash e': \theta$}
\BinaryInfC{$\pi \vdash ee' : \theta'$}
\DisplayProof
\end{center}

\begin{center}
\AxiomC{$\pi \vdash b  : \boolt$}
\AxiomC{$\pi \vdash e  : \theta$}
\AxiomC{$\pi \vdash e' : \theta$}
\TrinaryInfC{$\pi \vdash \cifthenelse{b}{e}{e'} : \theta$}
\DisplayProof
\end{center}

\end{block}

\end{frame}

\begin{frame}
\frametitle{Lenguaje $\lambdaarrow$}
\framesubtitle{Preliminares}

\begin{block}{Ejemplo de derivación de un juicio de tipado}
\begin{center}
\AxiomC{$\iota':\theta \in \iota:\intt, \iota':\realt$}
\UnaryInfC{$\iota:\intt, \iota':\realt \vdash \iota' : \theta$}
\UnaryInfC{$\iota:\intt \vdash \clambda{\iota'}{\realt} \ \iota' : \realt \rightarrow \theta$}
\UnaryInfC{$\vdash 
	\clambda{\iota}{\intt}{\clambda{\iota'}{\realt} \ \iota'} : 
									\intt \rightarrow \realt \rightarrow \theta$}
\DisplayProof
\end{center}
\end{block}

\end{frame}

\begin{frame}
\frametitle{Lenguaje $\lambdaarrow$}
\framesubtitle{Preliminares}

\begin{block}{Categorías para $\lambdaarrow$}

- $\Dom$, categoría de dominios y funciones continuas.\\
- $\Theta$, categoría discreta de $\lrangles{Type}$.\\
- $\Pi$, categoría discreta de $\lrangles{Context}$

\end{block}

\pause

\begin{block}{Funtores para $\lambdaarrow$}

$\semBrcks{ \ } : \Theta \rightarrow \Dom$ 
\quad \quad \quad \
$\semBrcks{ \ } : \Pi \rightarrow \Dom$
\quad \quad \quad
$\semBrcks{\pi \vdash e : \theta} : \semBrcks{\pi} \rightarrow \semBrcks{\theta}$\\
$\semBrcks{\delta}_0$ $=$ $(S_\delta)_\bot$
\quad \quad \quad \quad
$\semBrcks{\pi}_0$ $=$ $\prod\limits_{\iota \in dom \ \pi} \semBrcks{\pi\iota}$\\
$\semBrcks{\theta \rightarrow \theta'}_0$ $=$ $\semBrcks{\theta'}_0^{\semBrcks{\theta}_0}$

\end{block}

\end{frame}

\begin{frame}
\frametitle{Lenguaje $\lambdaarrow$}
\framesubtitle{Preliminares}

\begin{block}{Ecuaciones $\lambdaarrow$}
\begin{center}
$\semBrcks{\pi \vdash b : \boolt} \eta$ $=$ $\iotabot b$ \quad \quad \quad
$\semBrcks{\pi \vdash \neg e : \boolt}$ $=$ $\neg_{\odot} \circ \semBrcks{\pi \vdash e : \boolt}$
\end{center}

\begin{center}
$\semBrcks{\pi \vdash \iota : \theta}\eta$ $=$ $\eta \iota$
\end{center}

\begin{center}
$\semBrcks{\pi \vdash \clambda{\iota}{\theta}{e} : \theta \rightarrow \theta'}$ $=$
		$\semBrcks{\pi,\iota:\theta \vdash e : \theta'} \circ ext_{\iota,\theta}$
\end{center}

\begin{center}
$\semBrcks{\pi \vdash \rec{e} : \theta}\eta$ $=$ $\Y_{\theta} (\semBrcks{\pi \vdash e : \theta \rightarrow \theta}\eta)$
\end{center}

\begin{center}
$\semBrcks{\pi \vdash ee' : \theta'}\eta$ $=$ $\semBrcks{\pi \vdash e : \theta \rightarrow \theta'}\eta (\semBrcks{\pi \vdash e' : \theta}\eta)$
\end{center}

\end{block}

donde 
$ext_{\iota,\theta} : \semBrcks{\pi} 
					  \rightarrow \semBrcks{\theta} \rightarrow \semBrcks{\pi,\iota:\theta}$\\
\quad \quad \ \
$ext_{\iota,\theta} \eta \ x$ $=$ $[\ \eta \ | \ \iota:x \ ]$
\end{frame}

\begin{frame}
\frametitle{Lenguaje $\lambdaarrow$}
\framesubtitle{Continuidad}

\begin{block}{Teorema}
Dado un juicio de tipado $\pi \vdash e : \theta$ la ecuaci\'on sem\'antica
$\semBrcks{\pi \vdash e : \theta}$ es una funci\'on continua.
\end{block}

\pause

\begin{block}{}
Dada una cadena interesante
$\eta_0 \sqsubseteq \eta_1 \sqsubseteq \cdots$ en $\semBrcks{\pi}$, queremos ver

\begin{center}
$\semBrcks{\pi \vdash e : \theta}(\bigsqcup\limits^{\infty}_{i=0} \eta_i)$
$=$ $\bigsqcup\limits^{\infty}_{i=0} (\semBrcks{\pi \vdash e : \theta}\eta_i)$
\end{center}

\end{block}

\end{frame}

\begin{frame}
\frametitle{Lenguaje $\lambdaarrow$}
\framesubtitle{Continuidad}

\begin{block}{Ecuación semántica aplicación}
Si tenemos

\begin{center}
$\semBrcks{\pi \vdash ee' : \theta'} : \semBrcks{\pi} \rightarrow \semBrcks{\theta'}$\\
\end{center}

y definimos

\begin{center}
$f : \semBrcks{\pi} \times \semBrcks{\theta} \rightarrow \semBrcks{\theta'}$
\quad \quad \quad \quad \quad \
$g : \semBrcks{\pi} \rightarrow \semBrcks{\theta}$\\
$f = \uncurry (\semBrcks{\pi \vdash e : \theta \rightarrow \theta'})$
\quad
$g = \semBrcks{\pi \vdash e' : \theta}$
\end{center}

\end{block}

\end{frame}

\begin{frame}
\frametitle{Lenguaje $\lambdaarrow$}
\framesubtitle{Continuidad}

\begin{block}{Ecuación semántica aplicación}

\begin{diagram}
  \semBrcks{\theta'}^{\semBrcks{\theta}} \times \semBrcks{\theta} & \rTo^{\epsilon} & \semBrcks{\theta'}\\
  \uTo^{\widetilde{f} \times 1_{\semBrcks{\theta}}} & \ruTo^{f} \ruTo(2,4) & &  \\
  \semBrcks{\pi} \times \semBrcks{\theta} & & f \circ \lrangles{1_{\semBrcks{\pi}} , g} & \\
  \uTo^{\lrangles{1_{\semBrcks{\pi}} , g}}  \ruTo(3,2) & & &  \\
  \semBrcks{\pi} & & &
\end{diagram}

\end{block}

\end{frame}

\begin{frame}
\frametitle{Lenguaje $\lambdaarrow$}
\framesubtitle{Continuidad}

\begin{block}{Ecuación semántica aplicación}
Luego podemos definir

\begin{center}
$\semBrcks{\pi \vdash ee' : \theta'} : \semBrcks{\pi} \rightarrow \semBrcks{\theta'}$\\

\

$\semBrcks{\pi \vdash ee' : \theta'} = 
\epsilon \circ \lrangles{\semBrcks{\pi \vdash e : \theta \rightarrow \theta'}
						, \semBrcks{\pi \vdash e' : \theta}}$
\end{center}

además, tomando un $\eta$ en $\semBrcks{\pi}$\\

\

$\semBrcks{\pi \vdash ee' : \theta'}\eta = 
(\epsilon \circ \lrangles{\semBrcks{\pi \vdash e : \theta \rightarrow \theta'}
						, \semBrcks{\pi \vdash e' : \theta}})\eta$\\
\quad \quad \quad \quad \quad \quad 
$= \semBrcks{\pi \vdash e : \theta \rightarrow \theta'}\eta(\semBrcks{\pi \vdash e' : \theta}\eta)$

\end{block}

\end{frame}

\begin{frame}
\frametitle{Lenguaje $\lambdaarrow$}
\framesubtitle{Corrección regla-$\beta$}

\begin{block}{Nuevo juicio de tipado y su regla de inferencia}
\begin{center}
\AxiomC{$\pi \vdash \Delta\iota : \pi'\iota$ \ \ \ para todo $\iota$ en $dom(\pi)$}
\UnaryInfC{$\pi \vdash \Delta : \pi'$}
\DisplayProof

\end{center}

donde $\Delta : \lrangles{Id} \rightarrow\lrangles{Phrase}$
\end{block}

\pause

\begin{block}{Teorema Substitución}

Sean $\pi$ y $\pi'$ contextos, $\Delta$ una substituci\'on y supongamos 

\begin{center}
$\pi' \vdash e : \theta$ \quad $\pi \vdash \Delta : \pi'$
\end{center}

donde  
$\eta'\iota$ $=$ $\semBrcks{\pi \vdash \Delta\iota : \pi'\iota}\eta$
$\forall \iota \in \ dom \ \pi'$, entonces\\

\begin{center}
$\semBrcks{\pi' \vdash e : \theta}\eta'$ $=$ $\semBrcks{\pi \vdash e / \Delta : \theta}\eta$.
\end{center}

\end{block}

\end{frame}

\begin{frame}
\frametitle{Lenguaje $\lambdaarrow$}
\framesubtitle{Corrección regla-$\beta$}

\begin{block}{Teorema Corrección regla-$\beta$}
Dado un juicio de tipado $\pi \vdash (\clambda{\iota}{\theta}{e})e' : \theta'$ vale,

\begin{center}
$\semBrcks{\pi \vdash (\clambda{\iota}{\theta}{e})e' : \theta'}$ $=$
$\semBrcks{\pi \vdash e / \iota \leadsto e' : \theta'}$
\end{center}
\end{block}

\end{frame}

\section{Lenguaje $\lambdaleq$}

\begin{frame}
\frametitle{Lenguaje $\lambdaleq$}
\framesubtitle{Preliminares}

\begin{block}{Nuevo juicio de tipado}
Sean $\theta$, $\theta'$ tipos, diremos que $\theta$ es subtipo de $\theta'$ cuando

\begin{center}
$\theta \leq \theta'$
\end{center}

\end{block}

\pause

\begin{block}{Nuevas reglas de inferencia}



\end{block}

\end{frame}

\section{Lenguaje $\Alike$}

\begin{frame}
\frametitle{Lenguaje $\Alike$}
\framesubtitle{Preliminares}

\begin{block}

\end{block}

\end{frame}

\end{document}

